\documentclass{notes}

\newcommand{\cO}{\mathcal{O}}
\newcommand{\cG}{\mathcal{G}}
\newcommand{\B}{\vect{B}}
\newcommand{\cL}{\mathcal{L}}
\newcommand{\E}{\vect{E}}
\newcommand{\jc}{\vect{j}}
\newcommand{\rv}{\vect{r}}
\newcommand{\rhat}{\vect{\Hat{r}}}
\newcommand{\K}{\mathrm{K}}
\renewcommand{\Box}{\square}
\DeclareMathOperator{\curl}{curl}
\DeclareMathOperator{\dive}{div}
\DeclareMathOperator{\grad}{grad}

\begin{document}

\frontmatter

\title{Electrodynamics}
\lecturer{Dr.~M.~J.~Perry}
\maintainer{Paul Metcalfe}
\date{Mich\ae lmas 1997} \maketitle

\thispagestyle{empty}
\noindent\verb$Revision: 2.10 $\hfill\\
\noindent\verb$Date: 2001/11/01 09:52:07 $\hfill

\vspace{1.5in}

The following people have maintained these notes.

\begin{center}
\begin{tabular}{ r  l}
-- date & Paul Metcalfe
\end{tabular}
\end{center}

\tableofcontents

\chapter{Introduction}

These notes are based on the course ``Electrodynamics'' given by
Dr.~M.~J.~Perry in Cambridge in the Mich\ae lmas Term 1997.  The
recommended books for this course are discussed in the bibliography.

\alsoavailable
\archimcopyright

\mainmatter

\chapter{Point of departure}

This is a review of terminology and results from Special Relativity and
Electromagnetism (possibly rewritten in a more grown-up way).

\section{Maxwell's Equations}

These are :

\begin{align*}
\dive \E &= \frac{\rho}{\epsilon_0} \\
\curl \E &= - \dot{\B} \\
\dive \B &= 0 \\
\curl \B &= \mu_0 \jc + \mu_0 \epsilon_0 \dot{\E}.
\end{align*}

$\rho$ is the charge density.  $\epsilon_0$ is called the permittivity
of free space.  It is not a fundamental constant but merely determines
units.  Similarly, $\mu_0$ is the permeability of free space and
merely determines units.  $\mu_0$ and $\epsilon_0$ satisfy $\mu_0
\epsilon_0 = c^{-2}$, where $c$ is the speed of light (and a
fundamental constant).  In familiar units $c \approx 2.997 \times
10^8\ \mathrm{ms}^{-1}$, but we will choose units
such that $c = 1$.%
\footnote{Despite the fact that the Schedules mandate SI units.  Exam
questions will be set such that $c=1$.}
Dimensional analysis can replace $c$ in any derived formulae.

\section{Electrostatics}

This is the case where there is no current and a time independent
charge distribution.  Then Maxwell's equations reduce to $\dive \E =
\frac{\rho}{\epsilon_0}$ and $\curl \E = 0$.  We will assume $\B = 0$,
but it does not affect the equations for the electric field.

The electric field due to a point charge $q_1$ is $\E = \frac{q_1}{4 \pi
\epsilon_0} \frac{1}{r^2} \rhat$.  To measure the electric field we
can take a charge $q_2$ and measure the force $\vect{F} = q_2 \E$ on
it.

\subsection{Coulomb's Law}

The force between two point charges is
\[
\vect{F} = \frac{q_1 q_2}{4 \pi \epsilon_0} \frac{1}{r^2}
\]
directed on the line between the centres.  It is repulsive for two like
charges.

A point charge can be regarded as a charge distribution which is a
delta function: $\rho = q_1 \delta(\rv)$.  To find the electric field
due to a distribution of charges we can use linear superposition to
find $\E$ everywhere.  As $\curl \E = 0$ we can introduce the electrostatic
potential $\phi$ such that $\E = - \nabla \phi$.  Then $\nabla^2 \phi
= \frac{-\rho}{\epsilon_0}$.  We can solve this using a Green's function,
that is a function $G(\rv,\rv')$ such that $\nabla^2_r G(\rv,\rv') =
\delta(\rv,\rv')$. We can see that $G(\rv,\rv') = - \frac{1}{4 \pi}
\frac{1}{\abs{\rv - \rv'}}$.  Then
\[
\phi(\rv) = - \frac{1}{\epsilon_0} \int \ud^3 r'\, G(\rv,\rv') \rho(\rv').
\]

\begin{proof}
Firstly, we see that
\begin{align*}
\nabla^2_r \phi &= - \frac{1}{\epsilon_0} \int \ud^3 r'\, \nabla^2_r G(\rv,\rv')
\rho(\rv') \\
&= - \frac{1}{\epsilon_0} \int \ud^3 r'\, \delta(\rv,\rv') \rho(\rv') \\
&= \frac{-\rho(\rv)}{\epsilon_0}.
\end{align*}

Then we merely note that solutions to Poisson's equation are unique.
\end{proof}

\subsection{Multipole expansion}

Suppose we have a charge distribution in a region $B$ as shown.

\vspace{1in}

The multipole expansion of the potential is what happens to a general
expression for $\phi$ if $\abs{\rv} \gg \abs{\rv'}$.  We expand
$\frac{1}{\abs{\rv - \rv'}}$ using the binomial expansion.

\begin{align*}
\frac{1}{\abs{\rv - \rv'}} &= (r^2 - 2 r_i r_i' + r'{}^2)^{-\frac{1}{2}} \\
&= \frac{1}{r} \left(1 - \frac{2 r_i r_i}{r^2} + \frac{r'{}^2}{r^2}
\right)^{-\frac{1}{2}} \\
&= \frac{1}{r} \left(
1 + \frac{r_i r'_i}{r^2} - \frac{1}{2} \frac{r'{}^2}{r^2} + \frac{3}{2}
\frac{ \left( r_i r'_i\right)^2}{r^4} + \dots \right).
\end{align*}

We substitute into the general expression for $\phi$ to obtain
\begin{align*}
\phi(\rv) &= \frac{1}{4 \pi \epsilon_0 r}
\int_B \ud^3 r'\, \rho(\rv') \left(
1 + \frac{r_i r'_i}{r^2} + \frac{1}{2 r^4} \left(
3 r_i r_j r'_i r'_j - r^2 \delta_{ij} r'_i r'_j
\right)+ \dots \right).
\end{align*}

This is an expansion of $\phi$ in inverse powers of $r$.  The term in
$r^{-l}$ is referred to as the $2^l$-pole term.

When $l=0$ we have a monopole.  If $Q = \int \ud^3 r'\, \rho(\rv')$
then $\phi = \frac{Q}{4 \pi \epsilon_0 r}$, which is usually called the
Coulomb term.

When $l=1$ we have a dipole.  Let $d_i = \int d^3 r'\, \rho(\rv') r'_i$
(the dipole moment).  Then $\phi = \frac{1}{4 \pi \epsilon_0 r^3} r_i d_i$.

When $l=2$ we have a quadrupole and the contribution to $\phi$ is
$\frac{1}{4 \pi \epsilon_0 r^5} r_i r_j Q_{ij}$, where $Q_{ij}$ is
the quadrupole moment and $Q_{ij} = \frac{1}{2} \int \ud^3 r'\, \rho(\rv')
\left( 3 r'_i r'_j - \delta_{ij} r'{}^2 \right)$.  $Q_{ij}$ is a
symmetric tracefree tensor, as
\[
Q_{ij} \delta_{ij} = \frac{1}{2} \int \ud^3 r'\, \rho(\rv')
\left( 3 r'_i r'_j \delta_{ij} - \delta_{ij} \delta_{ij} r'{}^2 \right) = 0.
\]
It has 5 independent components.  In general the $r^{-l}$ term has
$2 l + 1$ independent components.  When $l=3$ we have the octopole moment
and when $l=4$ the hexadecapole moment, but these become increasingly
cumbersome.

\section{Special Relativity}

Special relativity has two postulates:

\begin{enumerate}
\item The laws of nature are the same in any inertial frame.
\item The speed of light is independent of the speed of its source.
\end{enumerate}

This leads us to consider Minkowski space, viz.
$x^\mu = (t,\vx) = (t,x^i)$.  $\mu$ runs from $0$ to $3$ and $i$ runs from
$1$ to $3$.  These are inertial co-ordinates.  If a particle is at rest
at $\vx = 0$ at $t=0$ then in remains at rest at $\vx = 0$ for all time.
$t$ is then the proper time for that particle --- what a clock sitting on
the particle would measure.

We can relate the physics in one inertial frame to another by Lorentz
transformations.  Suppose that one has a second frame moving with
velocity $v$ in the $x$ direction relative to the first frame.  Then
we have new inertial co-ordinates

\begin{align*}
t' &= \gamma(v) (t - v x) \\
x' &= \gamma(v) (x - v t) \\
y' &= y \\
z' &= z,
\end{align*}

where $\gamma(v) = \frac{1}{\sqrt{1-v^2}}$.  This can be written
$x'{}^\mu = \Lambda^\mu{}_\nu x^\nu$, where $\Lambda^\mu{}_\nu$ is the
matrix form of the Lorentz transformation, in this case:

\[
\begin{pmatrix}
\gamma & - \gamma v & 0 & 0 \\
- \gamma v & \gamma & 0 & 0 \\
0 & 0 & 1 & 0 \\
0 & 0 & 0 & 1
\end{pmatrix}
\]

To find the Lorentz transformation of arbitrary motion described by
a unit vector $\vect{n}$ then we consider
$R^{-1}(\vect{n}) \Lambda R(\vect{n})$, where $R(\vect{n})$ is a rotation
to move $\vect{n}$ into the $x$ direction.
\[
R = \begin{pmatrix}
1 & 0 & 0 & 0 \\
0 &   &   &   \\
0 &   & R_3 & \\
0 &   &   &   \\
\end{pmatrix}
\]
with $R_3$ an ordinary spatial rotation.  We can make Lorentz
transformations look like rotations --- if we put $\gamma(v) = \cosh \phi$,
then $v \gamma(v) = \sinh \phi$, which implies that $v = \tanh \phi$.  This
is sometimes called a hyperbolic rotation.

We define a distance $\ud s^2 = -\ud t^2 + \ud x^2 + \ud y^2 + \ud z^2$
between two
infinitesimally separated points.  This can be shown to be invariant under
Lorentz transforms.  If $\ud t = 0$ then it reduces to the ordinary Euclidean
metric on $\R^3$.

If $\ud s^2 > 0$ we say the two points are spacelike separated, if $\ud s^2 < 0$
they are timelike separated and if $\ud s^2 = 0$ they are null, or lightlike
separated.  The interior of the light cone has $\ud s^2 < 0$ and the exterior
$\ud s^2 > 0$.

Proper distance is defined for spacelike separated events to be $\ud s$
and proper time $\ud\tau$ for timelike separated events by $\ud\tau^2
= -\ud s^2$.
The invariance of $\ud\tau$ means that the time seen by a clock sitting on
some object can be computed in any inertial frame.

It is easier to write $\ud s^2 = \eta_{\mu \nu} \ud x^\mu \ud x^\nu$, where
$\eta_{\mu \nu}$ is called the metric tensor and is
\[
\begin{pmatrix}
-1 & 0 & 0 & 0 \\
0  & 1 & 0 & 0 \\
0  & 0 & 1 & 0 \\
0  & 0 & 0 & 1
\end{pmatrix}.
\]

We also have the inverse metric $\eta^{\mu \nu}$ defined by
$\eta^{\mu \nu} \eta_{\nu \rho} = \delta^\mu_\rho$.  The matrix form
of $\eta^{\mu \nu}$ is (obviously?)
\[
\begin{pmatrix}
-1 & 0 & 0 & 0 \\
0 & 1 & 0 & 0 \\
0 & 0 & 1 & 0 \\
0 & 0 & 0 & 1
\end{pmatrix}.
\]

Given two vectors $x^\mu$ and $y^\mu$ we want a scalar product that should
be invariant under Lorentz transformations, and $S = \eta_{\mu \nu}
x^\mu y^\nu$ will do the trick.  This gives us the idea of defining covectors
by $x_\nu = \eta_{\nu \mu} x^\mu$ and then $S = x_\nu y^\nu$.  Then
there is the inverse operation
$\eta^{\mu \nu} x_\nu = \eta^{\mu \nu} \eta_{\nu \rho} x^\rho = x^\mu$.

We ask ourselves how a covector transforms, and we obtain
$x_\nu \mapsto x'_\nu = \Lambda_\nu{}^\mu x_\mu$, where
$\Lambda_\nu{}^\mu = \eta_{\nu \rho} \Lambda^\rho{}_\sigma \eta^{\sigma \mu}$,
or in matrix form

\[
\begin{pmatrix}
\gamma & \gamma v & 0 & 0 \\
\gamma v & \gamma & 0 & 0 \\
0 & 0 & 1 & 0 \\
0 & 0 & 0 & 1
\end{pmatrix}
\]

for the transform we looked at earlier.  We see that
$(\Lambda^\mu{}_\nu)^{-1} = \Lambda_\nu{}^\mu$ and
\begin{align*}
x_\mu y^\mu \mapsto x'_\mu y'{}^\mu &= \Lambda_\mu{}^\rho
x_\rho \Lambda^\mu{}_\sigma y^\sigma \\
&= \Lambda_\mu{}^\rho \Lambda^\mu{}_\sigma x_\rho y^\sigma \\
&= \delta^\rho_\sigma x_\rho y^\sigma \\
&= x_\sigma y^\sigma,
\end{align*}
which is, on the whole, a good thing.  This is analogous to rotations in
$\R^3$ preserving the metric, the rotations being classified by
$R^T R = 1$ and $R_{ij} \delta_{jk} R_{kl} = \delta_{il}$ excluding
reflections.  A Lorentz transform preserves the Minkowski space metric and
$\Lambda_\sigma{}^\mu \Lambda_\tau{}^\nu \eta_{\mu \nu} = \eta_{\sigma \tau}$.
To see the equivalence multiply both sides by $\eta^{\tau \lambda}$ to
get (eventually) $\delta^\lambda_\sigma = \delta^\lambda_\sigma$.

The Lorentz transforms are defined by $\Lambda_\mu{}^\nu
\Lambda_\sigma{}^\tau \eta_{\nu \tau} = \eta_{\mu \sigma}$ (the group
$SO(3,1)$) with no spatial reflections and preservation of time,
giving the Lorentz group.

\chapter[Relativistic Electromagnetism]%
{The relativistic theory of electromagnetism}

We start with the Lorentz force law, $\vect{F} = e \E + e \vect{v} \wedge
\B$ and seek to generalise it.  Non-relativistically we have
\[
m \diff{^2x^i}{t^2} = e \left( \E + \vect{v} \wedge \B \right)^i,
\]
and we already know the equation of motion for a free relativistic particle,
\[
m \diff{^2x^\mu}{\tau^2} = 0.
\]

We also recall the non-relativistic velocity 4-vector,
\[
\diff{x^\mu}{t} = u^\mu = (1,\vect{u}),
\]
and we know that $\ud \tau^2$ is Lorentz invariant and hence
$-1 = u^\mu u^\nu \eta_{\mu \nu}$.  We guess a force law
\[
m \diff{^2 x^\mu}{\tau^2} = e F^\mu{}_\nu \diff{x^\nu}{\tau}.
\]

We see immediately that by the quotient theorem, $F^\mu{}_\nu$ must be
a tensor.  We also know that this equation must be true in any inertial
frame, and so is always true.  We take the non-relativistic case,
where $\diff{}{\tau} = \diff{}{t}$ to find what $F^\mu{}_\nu$ must be.
\[
e E_x + e ( v_y B_z - v_z B_y)
= e F^1{}_0 \diff{t}{\tau} + e F^1{}_1 \diff{x}{\tau}
+ e F^1{}_2 \diff{y}{\tau} + e F^1{}_3 \diff{z}{\tau}
\]
We thus identify $F^1_{0} = E_x$, $F^1{}_1 = 0$, $F^1{}_2 = B_z$ and
$F^1{}_3 = - B_y$.  We repeat this process with $y$ and $z$, and
lower the index, giving us
\[
\eta_{\mu \lambda} F^\lambda{}_\nu
= F_{\mu \nu}
= \begin{pmatrix}
 & & & \\
E_x & 0 & B_z & -B_y \\
E_y & -B_z & 0 & B_x \\
E_z & B_y & - B_x & 0
\end{pmatrix}.
\]
We note that the spatial part of this is antisymmetric, and since we
treat space and time equivalently, we can finally define
\[
F_{\mu \nu}
= \begin{pmatrix}
0 & - E_x & - E_y & - E_z \\
E_x & 0 & B_z & -B_y \\
E_y & -B_z & 0 & B_x \\
E_z & B_y & - B_x & 0
\end{pmatrix}.
\]
This is the electromagnetic field (strength) tensor, or Maxwell tensor.
The only puzzle is what the time component of the relativistic Lorentz
equation represents.  It is $m \diff{^2 t}{\tau^2} = e F^0{}_i
\diff{x^i}{\tau}$.  We note that $u^\mu = \gamma (1,\vect{v})$
and that $\gamma$ is a kind of relativistic energy, giving
$\diff{}{\tau} \left( m \gamma \right) = e \vect{v} . \E$, which we
know as ``the rate of change of energy equals the rate of doing work
by the electric field''.

\subsection{Relativistic motion in constant electric field}

We consider a constant electric field in the $x$ direction.  We
have a particle with charge $e$ which starts at rest at the origin.  The
non-relativistic case is $m \ddot{x} = e E$, which gives
$x = \frac{e E}{2 m} t^2$ and $\dot{x} = \frac{e E}{m} t$, which eventually
exceeds $1$!

Relativistically $m \ddot{y} = m \ddot{z} = 0$, which are trivial.  We also
have the equations $m \ddot{x} = e F^x{}_0 \dot{t} = e E \dot{t}$
and $m \ddot{t} = e E \dot{x}$.

\marginpar{\framebox{%
\parbox{1in}{In the relativistic case, $\dot{f} \equiv \diff{f}{\tau}$.}}}

We integrate these once and use the initial conditions to get
$m \dot{t} = e E x + C_1$ and $m \dot{x} = e E t$.  (Note that we
set $\tau = 0$ at $t = 0$.)  Integrating again we get
\begin{gather*}
x(\tau) = A \sinh \frac{e E \tau}{m} + B \cosh \frac{e E \tau}{m}
- \frac{m C_1}{e E} \\
t(\tau) = A \cosh \frac{e E \tau}{m} + B \sinh \frac{e E \tau}{m}
+ \Tilde{C}.
\end{gather*}

We have the boundary conditions $x = \dot{x} = 0$ at $\tau = 0$ and
$t = 0$ at $\tau = 0$.  There is a temptation to put $\dot{t} = 0$
at $\tau = 0$ --- but this is inconsistent, as
$-1 = u^\mu u_\mu = (-\dot{t}^2 + \dot{x}^2)$.  Thus we put $\dot{t} = 1$
at $\tau = 0$ --- which we could have guessed, we know that at rest,
co-ordinate time is the same as proper time.  Finally, we get
\begin{gather*}
x(\tau) = \frac{m}{e E}\left( \cosh \frac{e E \tau}{m} - 1 \right) \\
t(\tau) = \frac{m}{e E} \sinh \frac{e E \tau}{m}.
\end{gather*}

To find the velocity we can write $x(t)$ by eliminating $\tau$
\[
x(t) = \frac{m}{e E} \left( \left( 1 + \frac{e^2 E^2 t^2}{m^2}
\right)^{\frac{1}{2}} - 1 \right).
\]

We can now find the velocity
\[
\diff{x}{t} = \frac{e E}{m}\frac{t}{\sqrt{1 + \frac{e^2 E^2 t^2}{m}}}.
\]

For small $t$ we have the reassuring $\diff{x}{t} = \frac{e E t}{m}$,
but in the large $t$ case we find only that $v \to 1$ from below, and
is always less than $1$.

\section{Transformation of $F_{\mu \nu}$}

We Lorentz transform (in the $x$ direction with velocity $v$) the
tensor $F_{\mu \nu}$ to see how the electric and magnetic fields
change under Lorentz transformations.

We know that $F_{\mu \nu} \mapsto F'_{\mu \nu}
= \Lambda_\mu{}^\rho \Lambda_\nu{}^\sigma F_{\rho \sigma}$, and
we just perform these sums to get:
\begin{align*}
E_x' &= E_x \\
E_y' &= \gamma ( E_y - v B_z ) \\
E_z' &= \gamma ( E_z + v B_y ) \\
B_x' &= B_x \\
B_y' &= \gamma (B_y + v E_z) \\
B_z' &= \gamma (B_z - v E_y).
\end{align*}

These are radically different from what we would expect if there were two
electric and magnetic 4-vectors.

\section{Lorentz invariant scalars}

We know that $F_{\mu \nu}$ and $\eta_{\mu \nu}$ are Lorentz invariant, and
we can derive some Lorentz scalars from them.  The most obvious one
is $F_{\mu \nu} \eta^{\mu \nu}$, but as $F$ is antisymmetric and $\eta$
symmetric this evaluates to zero.  A more useful Lorentz scalar
is $F_{\mu \nu} F^{\mu \nu} = 2 \left(\B^2 -\E^2 \right)$.

We can get another Lorentz scalar by introducing the \emph{alternating
tensor}, which is defined as

\[
\varepsilon^{\mu \nu \rho \sigma}
= \begin{cases} 1 & \text{if $\mu \nu \rho \sigma$ is an even permutation
of $0 1 2 3$} \\
-1 & \text{if $\mu \nu \rho \sigma$ is an odd permutation of $0 1 2 3$} \\
0 & \text{otherwise.}
 \end{cases}
\]

We can now define the dual field strength tensor, $G^{\mu \nu}
= \frac{1}{2} \varepsilon^{\mu \nu \rho \sigma} F_{\rho \sigma}$.  We
can evaluate this
\[
G_{\mu \nu} =
\begin{pmatrix}
0 & -B_x & - B_y & -B_z \\
B_x & 0 & -E_z & E_y \\
B_y & E_z & 0 & -E_x \\
B_z & -E_y & E_x & 0
\end{pmatrix}.
\]

The dual tensor $G$ can be found from $F$ by $\E \mapsto \B$ and
$\B \mapsto - \E$, which is sometimes called a duality rotation.  We
can now define a further Lorentz scalar $F_{\mu \nu} G^{\mu \nu}
= - 4\, \E . \B$.

\section{Tensorial form of Maxwell's equations}

\marginpar{%
\framebox{%
$\partial_i \equiv \pd{}{x^i}$ and $\partial_\mu \equiv
\pd{}{x^\mu}$}}
We start from Maxwell's equations to see what they turn into in tensor
notation.  We take them in an slightly unusual order, $\dive \E =
\mu_0 \rho$ and $\curl \B = \mu_0 \jc + \dot{\E}$ and seek to write them
as tensor equations.

We note that $E_i = F_{i 0} = -F^{0 i}$ and $B_i = \frac{1}{2} \epsilon_{ijk}
F^{jk}$.  The divergence equation becomes $\partial_i F_{i 0} = \mu_0 \rho$.
The curl equation is
\begin{align*}
\frac{1}{2} \epsilon_{i j k} \partial_j \epsilon_{k l m} F^{l m}
& = \frac{1}{2} \left( \partial_j F_{i j} - \partial_j F_{j i} \right)\\
& = \partial_j F_{ij} = \mu_0 j_i + \partial_0 F_{i 0}.
\end{align*}

We hope that this is consistent with a tensor equation
$\partial_\mu F^{\mu \nu} = X^\nu$.  If we study this we see that
this works if $X^\mu= - \mu_0 ( \rho, \jc )$.  Thus two of the Maxwell
equations become
\[
\partial_\mu F^{\mu \nu} = - \mu_0 j^\nu.
\]
where $j^\mu = (\rho,\jc)$ is the 4-vector (electric) current.  This
illustrates that a moving charge and a current are just the same thing as
$j^\nu$ is a four-vector and so is consistent with Lorentz transforms.

Incidentally, we have not lost conservation of charge as
$0 \equiv \partial_\nu \partial_\mu F^{\mu \nu} = - \mu_0 \partial_\nu
j^\nu$.

We now go after the next two Maxwell equations, $\dive \B = 0$
and $\curl \E = - \dot{\B}$.  The first is easy; it gives
$\partial_i \epsilon_{i j k} F_{j k} = 0$.  We guess that this is a
component of $\varepsilon^{\mu \nu \rho \sigma} \partial_\nu
F_{\rho \sigma} = 0$, and indeed if we evaluate the spatial components
we reproduce the last Maxwell equation. 

We can rewrite this in terms of the dual field-strength tensor to get
$\partial_\mu G^{\mu \nu} = 0$ --- this says that there is no magnetic
current.  It turns out to be more useful to explicitly antisymmetrize our
equation to get
\[
\partial_\nu F_{\rho \sigma} + \partial_\rho F_{\sigma\nu}
+ \partial_\sigma F_{\nu \rho} = 0.
\]

\subsection{Potentials}

In the non-relativistic case we know that $\E$ and $\B$ can be derived
from potentials:

\begin{align*}
\E &= -\grad \phi - \dot{\vect{A}} \\
\B &= \curl \vect{A}.
\end{align*}

It turns out that we can come up with an electromagnetic 4-vector potential,
$A^\mu = (\phi, \vect{A})$ such that $F_{\mu \nu} = \partial_\mu A_\nu -
\partial_\nu A_\mu$ (simply expand this to see).

We know that $\vect{A}$ is not unique in non-relativistic electromagnetism;
we can add on the gradient of any scalar function --- called a gauge
transformation.  Similarly, we see that the 4-vector potential $A_\mu$ is
unique up to $\partial_\mu \Lambda$ for any scalar function $\Lambda$.  This
means we can try to impose extra conditions on $A_\mu$ which (partially)
prevents gauge transformations (called gauge fixing).  The point is to ensure
that a given $F_{\mu \nu}$ is the product of a moderately unique
$A_\mu$.\footnote{What ``moderately unique'' means has yet to be defined.}

A useful covariant gauge is to impose $\partial_\mu A^\mu = 0$.  Thus
if we have $A^\mu$ such that $F_{\mu \nu} = \partial_\mu A_\nu
- \partial_\nu A_\mu$ and we set $A'{}^\mu = A^\mu + \partial^\mu \Lambda$,
then we can have $0 = \partial_\mu A'{}^\mu = \partial_\mu A^\mu
+ \partial_\mu \partial^\mu \Lambda$.  In principle we can solve
$\partial_\mu \partial^\mu \Lambda = - \partial_\mu A^\mu$, so this
gauge condition is possible.

If we consider this a little more, we see that $A^\mu$ is still non-unique,
but only up to a solution of the wave equation, $\partial_\mu \partial^\mu
\Lambda = 0$.  Solutions of the wave equation are a combination of
plane waves, $e^{\imath k_\mu x^\mu}$, where $k^\mu = (\omega,\vect{k})$
and $\omega = \abs{\vect{k}}$.  $k^\mu$ is the wave 4-vector or the momentum
4-vector.

If we wish to find $F^{\mu \nu}$ given a collection of charges and
currents we solve the equation $\partial_\mu \partial^\mu A^\nu -
\partial_\mu \partial^\nu A^\mu = - \mu_0 j^\nu$.  This is where our
gauge condition comes in useful, we get $\partial_\mu \partial^\mu
A^\nu = - \mu_0 j^\nu$, or $\Box A^\nu = - \mu_0 j^\nu$, where
$\partial_\mu \partial^\mu \equiv \Box$ (pronounced ``box'').  We can
now see that $\partial_\nu A^\nu = 0$ is sensible --- it is compatible
with current conservation.

\section{Least action principles}

\subsection{Particle motion}

For a single free particle in special relativity we have the action
\[
I = \int \ud \tau\, m \sqrt{- \dot{x}^\mu \dot{x}^\nu \eta_{\mu \nu}}.
\]

The usual Euler-Lagrange equations
$\diff{}{\tau} \left( \pd{\cL}{\dot{x}^\mu} \right) - \pd{\cL}{x^\mu} = 0$ give
$- m \ddot{x}_\mu = 0$.

For a particle with charge $e$ in a potential $A^\mu$ we can generalise
this to
\[
I = \int \ud \tau\, m \sqrt{- \dot{x}^\mu \dot{x}^\nu \eta_{\mu \nu}}
- e A_\mu \dot{x}^\mu.
\]

But can this possibly be gauge invariant --- the quantity $A_\mu$
appears explicitly?  Suppose we make a small gauge transformation
such that $\delta A_\mu = \partial_\mu \Lambda$.  Then
\begin{align*}
\cL &\to \cL - e \dot{x}^\mu \partial_\mu \Lambda \\
&= \cL - \diff{}{\tau} e \Lambda.
\end{align*}

Therefore the Lagrangian only changes by a total derivative - which
does not affect the equations of motion.

This is all somewhat academic if varying $x^\mu$ does not give us the Lorentz
force law.  Using the Euler-Lagrange equations we get
\begin{align*}
0 &= \diff{}{\tau} \left( \frac{- m \dot{x}^\nu \eta_{\mu \nu}}
{\sqrt{- \dot{x}^\mu \dot{x}^\nu \eta_{\mu \nu}}} \right)
- \diff{}{\tau} \left(e A_\mu \right) + e \dot{x}^\nu \partial_\mu A_\nu \\
\intertext{$\tau$ is the proper time so $\dot{x}^\mu \dot{x}^\nu \eta_{\mu \nu}
= -1$ on the worldline}
&= \diff{}{\tau} \left(- m \dot{x}_\mu\right)  - e \dot{x}^\nu
\partial_\nu A_\mu + e \dot{x}^\nu \partial_\mu A_\nu \\
&= -m \ddot{x}_\mu + e \dot{x}^\nu F_{\mu \nu}.
\end{align*}

This is the Lorentz force law.  One can use the action as a quick way of
finding the motion of a particle.  For a constant electric field in
the $x$ direction we get $A^x = - E t$ and all other components of
$A^\mu$ are zero.  To get the motion of the particle we vary the action
\[
I = \int \ud \tau\, m \sqrt{\dot{t}^2 - \dot{x}^2} + e E t \dot{x}.
\]

The Euler-Lagrange equations give the same differential equations for the
motion of the particle as before, but more easily.

\subsection{Field action}

We also want to find a Lagrangian $\cL$ that reproduces Maxwell's equations,
that is if we take
\[
I = \int \ud^4 x\, \cL,
\]
$\delta I = 0$ under variations of something or other must
reproduce Maxwell's equations.

$\cL$ must be a Lorentz scalar, built out of $F_{\mu \nu}$ (or $A_\mu$)
and it must be gauge invariant.  The Maxwell equations involve first
derivatives of $F_{\mu \nu}$ (or second derivatives of $A_\mu$).  So the
only real possibilities are varying $A_\mu$ and $\cL$ must be quadratic
in $F_{\mu \nu}$.  $\varepsilon^{\mu \nu \rho \sigma} F_{\rho \sigma}$
gives nothing.  The usual choice is
\[
\cL = - \frac{1}{4 \mu_0} F^{\mu \nu} F_{\mu \nu} + j_{\mu} A^\mu.
\]

This is gauge invariant.  If $\delta A_\mu = \partial_\mu \Lambda$, then if
we assume $F_{\mu \nu} = \partial_\mu A_\nu - \partial_\nu A_\mu$ we get
$\delta F_{\mu \nu} = \partial_\mu \partial_\nu \Lambda - \partial_\nu
\partial_\mu \Lambda = 0$.

Now consider the $\int j_\nu A^\nu$ part, which is thus the only part
of the action which can potentially cause problems.  When $A_\mu
\mapsto A_\mu + \partial_\mu \Lambda$,
\begin{align*}
I &\mapsto I + \int \ud^4 x\, j_\nu \partial^\nu \Lambda \\
&=I + \int \ud^4 x\, \left(\partial^\nu j_\nu\right) \Lambda \\
&=I \qquad \text{using conservation of charge.}
\end{align*}

Thus the action is gauge invariant.

\begin{align*}
I &= - \int \ud^4 x \, \left(\frac{1}{4 \mu_0} F_{\mu \nu} F^{\mu \nu} -
j_\mu A^\mu \right) \qquad \text{and} \\
\delta I &= - \int \ud^4 x \, \left(\frac{1}{4 \mu_0}
\left( \delta F_{\mu \nu} F^{\mu \nu} + F_{\mu \nu} \delta F^{\mu \nu}
\right) - j^\mu \delta A_\mu \right) \\
&= - \int \ud^4 x \, \left(\frac{1}{2 \mu_0} \delta F_{\mu \nu} F^{\mu \nu}
- j^\nu \delta A_\nu \right)\\
&= - \int \ud^4 x \, \left( \frac{1}{2 \mu_0} \left(
\partial_\mu \delta A_\nu - \partial_\nu \delta A_\mu
\right) F^{\mu \nu}
- j^\nu \delta A_\nu \right) \\
&= - \int \ud^4 x \, \left(
\frac{1}{\mu_0} \left( \partial_\mu \delta A_\nu \right) F^{\mu \nu}
- \delta A_\nu j^\nu \right) \\
&= \int \ud^4 x \, \delta A_\nu \left( \tfrac{1}{\mu_0} \partial_\mu
F^{\mu \nu} + j^\nu \right).
\end{align*}

We perform the last line by integrating by parts and assuming that
boundary conditions are all zero.  As we can arbitrarily vary
$A_\mu$ we must have $\tfrac{1}{\mu_0} \partial_\mu F^{\mu \nu} = - j^\nu$.
The other Maxwell equation is automatic as we have assumed that
$F_{\mu \nu}$ is derived from a potential $A_\mu$. This least action
principle requires that there is no magnetic current.

\chapter{Energy - Momentum Tensor}

\section{Definition}

We seek a relativistic form of the field energy.  We define the
(stress-)energy tensor
\[
T^{\mu \nu} = \frac{1}{\mu_0} \left( F^{\mu \sigma} F^\nu{}_\sigma
- \tfrac{1}{4} \eta^{\mu \nu} F^{\rho \sigma} F_{\rho \sigma}
\right).
\]

We note in passing that this is a symmetric tensor. Note that the
$T^{0 0}$ component is $\frac{1}{2 \mu_0} \left( E^2 + B^2 \right)$,
which reproduces the non-relativistic energy density.  To interpret
the rest of the components of $T^{\mu \nu}$ we recall Poynting's
theorem.

\textbf{Poynting's Theorem.}   Let $D$ be a region in space.  Then the
rate of change of energy in $D$ is

\begin{align*}
\frac{1}{2 \mu_0} \pd{}{t} \int_D \E^2 + \B^2\, \ud V
&= \frac{1}{\mu_0} \int_D \E.\dot{\E} + \B.\dot{\B}\, \ud V \\
&= \frac{1}{\mu_0} \int_D \E.\left( \curl \B - \mu_0 \jc \right)
- \B.\curl \E\, \ud V \\
&= - \int_D \jc . \E\, \ud V
+ \frac{1}{\mu_0} \int_D \E.\curl \B  - \B.\curl \E\, \ud V \\
&= - \int_D \jc . \E\, \ud V - \int_{\partial D} \vect{N}.\ud \vect{S},
\end{align*}
where we have introduced the Poynting vector $\vect{N} = \frac{1}{\mu_0}
\E \wedge \B$.  The Poynting vector is the energy flux.

By performing the sum we find that $T^{0 k} = N^k$.  We find that

\[
T^{\mu \nu}
= \left( \text{\small\begin{tabular}{c | c}
  Energy & Energy flux \\
  density & \\ \hline
  Energy & \\
  flux   & ``Stress'' \\
         &
\end{tabular}}\right).
\]

We want to evaluate the ``stress'' part of this tensor.

\begin{align*}
T_{ij} &= \tfrac{1}{\mu_0} \left( F_{i \mu} F_j{}^\mu - \tfrac{1}{4}
\eta_{ij} F^{\rho \sigma} F_{\rho \sigma}\right) \\
&= \tfrac{1}{\mu_0} \left( F_{i 0} F_j{}^0 + F_{i k} F_j{}^k
- \tfrac{1}{4} \delta_{i j} \left( -2 E^2 + 2 B^2\right) \right) \\
&= \tfrac{1}{\mu_0} \left( -E_i E_j + \epsilon_{i k l} B_l \epsilon_{j k m}
B_m + \tfrac{1}{2} \delta_{i j} E^2 - \tfrac{1}{2} \delta_{i j} B^2 \right)\\
&= \tfrac{1}{\mu_0} \left( -E_i E_j + \tfrac{1}{2} \delta_{i j} E^2
- B_i B_j + \tfrac{1}{2} \delta_{i j} B^2 \right).
\end{align*}

This is the Maxwell stress tensor, and can be thought of as the pressure
of the electromagnetic field.

\subsection{Conservation of energy-momentum}

We compute the divergence of $T^{\mu \nu}$.

\begin{align*}
\partial_\nu T^{\mu \nu} &= \tfrac{1}{\mu_0}
\partial_\nu \left( F^{\mu \sigma} F^\nu{}_\sigma
- \tfrac{1}{4} \eta^{\mu \nu} F^{\rho \sigma} F_{\rho \sigma} \right) \\
&= \tfrac{1}{\mu_0} \left(
\left( \partial_\nu F^{\mu \sigma} \right) F^\nu{}_\sigma
+ F^{\mu\sigma} \left( \partial_\nu F^\nu{}_\sigma \right)
- \tfrac{1}{2} \left(\partial^\mu F^{\rho \sigma} \right) F_{\rho \sigma}
\right) \\
&= -F^{\mu \sigma} j_\sigma  + \tfrac{1}{\mu_0} F_{\rho \sigma} \left(
-\tfrac{1}{2} \partial^\mu F^{\rho \sigma} + \partial^\rho F^{\mu \sigma}
\right) \\
&= -F^{\mu \sigma} j_\sigma  - \tfrac{1}{2 \mu_0} F_{\rho \sigma} \left(
\partial^\mu F^{\rho \sigma} + \partial^\sigma F^{\mu \rho}
+ \partial^\rho F^{\sigma \mu} \right)\\
&= -F^{\mu \sigma} j_\sigma \qquad \text{by Maxwell's equations.}
\end{align*}

Thus in the absence of charges/currents, the energy-momentum tensor is
conserved.  We can evaluate the right hand side of this equation to get
\[
\partial_\nu T^{\mu \nu} = \left( - \jc.\E, \rho \E + \jc \wedge \B \right).
\]

The time component of this is the work done by the electromagnetic field
and the spatial components give the electric force on a current $\jc$
due to $\B$ and on a charge density $\rho$ due to $\E$.

\section{Plane waves}

This has the equation $\Box A^\mu = 0$ in the gauge $\partial_\mu
A^\mu = 0$, which has solutions
\[
A_\mu = A \epsilon_\mu \exp \left( \imath k_\sigma x^\sigma \right),
\]

Note that this is a complex solution, so when we work with $\E$ and $\B$
we must take the real part of the \emph{field}, which corresponds to the
imaginary part of $A_\mu$.

$\epsilon_\mu$ is the polarisation vector, and $k_\sigma = \left( -\omega,
\vect{k}\right)$ is the wave 4-vector.  $\omega$ is the angular frequency,
$\vect{k}$ is the wave vector and $\abs{\vect{k}} = \omega$.  $A$ is
the amplitude.  Imposing the gauge condition requires $\epsilon_\mu k^\mu =0$,
so we have the transversality of the wave.

This does not completely specify the gauge.  If we let $\delta A_\mu
= \partial_\mu \Lambda$ with $\Lambda = - \imath C e^{\imath k.x}$ then
$\epsilon_\mu \mapsto \epsilon_\mu + C k_\mu$. Since $k_\mu k^\mu = 0$ this
preserves the gauge condition.  This freedom is usually exploited to put
$A^0 = 0$.  In this case,
\[
A_\mu = (0,A \vect{\epsilon} e^{\imath k.x}) = (0,A \vect{\epsilon}
e^{\imath ( \vect{k}.\vx - \omega t )}),
\]
where $\vect{\epsilon}$ is a spatial vector.  The gauge condition gives
$\vect{k}.\vect{\epsilon} = 0$.

Finding $\E$ and $\B$ from $A_\mu$ is easy:
\[
E_i = -F_{0 i} = -\partial_0 A_i + \partial_i A_0
= \imath \omega A_i.
\]
and
\[
B_i = \tfrac{1}{2} \epsilon_{i j k} F_{j k}
= \epsilon_{i j k} \partial_j A_k 
= - \imath \vect{k} \wedge \vect{A}.
\]

The physical fields correspond to the real parts of these quantities.

\section{Radiation pressure}

\vspace{1in}

Suppose that we have a situation as drawn above, with a plane wave
propagating in the $z$ direction.  Then the electric and
magnetic fields are in the $y$ and $x$ directions respectively, with
$E_y = \omega A \sin \omega (t-z)$ and $B_x = \omega A \cos \omega (t-z)$.

The rate of flow of momentum per unit area is $\abs{\vect{N}}$, where
$\vect{N}$ is the Poynting vector and this has a time average
$\tfrac{1}{2 \mu_0} \omega^2 A^2$.  This is coincidentally the same as the
energy density.  We also evaluate the stress-energy tensor
\[
T_{i j} = \tfrac{1}{\mu_0} \left( \tfrac{1}{2} \delta_{i j} \left(E^2
+B^2 \right) - E_i E_j - B_i B_j\right).
\]

This is clearly symmetric, and evaluating the diagonal components we get
$\langle T_{xx} \rangle = 0$, $\langle T_{yy} \rangle = 0$ and
$\langle T_{zz} \rangle = \tfrac{1}{2 \mu_0} \omega^2 A^2$.  There is
a pressure due to the wave, but importantly \emph{it is not isotropic}.

\chapter{Solving Maxwell's Equations}

\section{A Green's Function}

We hope to find a general expression for $A_\mu$ given a
time dependent distribution of charges and currents.  We will work
in the $\partial_\mu A^\mu = 0$ gauge, and so we have to
solve the equation $\Box A^\mu = - \mu_0 j^\mu$.  We proceed naively
and see what happens.

We hope to find a Green's function $G(x,x')$ such that
$\Box G = \delta^{(4)} (x,x')$ and so
\[
A^\mu(x) = - \mu_0 \int \ud^4 x'\, G(x,x') j^\mu(x')
\]
is a solution of $\Box A^\mu = - \mu_0 j^\mu$.  One problem is
that $\Box$ is a hyperbolic operator so there exist non-trivial solutions
to $\Box \phi = 0$ with $\phi \not \to 0$ at infinity.

The four dimension Fourier transform is defined by

\[
\Hat{f}(k) = \int \ud^4 x\, f(x) e^{-\imath k.x}.
\]

The minus sign in the exponential is not arbitrary.  If $f$ is a plane
wave $f(x) \sim e^{\imath p.x}$ then
$\Hat{f}(k) = (2 \pi)^4 \delta^{(4)}(p-k)$, which is what we want.

We will solve $\Box G(x,x') = \delta^{(4)}(x,x')$ using the Fourier
transform.  $\Hat{G}(k,x') = - k^{-2} e^{- \imath k.x}$ and so defining
$z^\mu = x^\mu - x'{}^\mu = (z^0,\vect{z})$ we find
\[
G(x,x') = \frac{1}{(2 \pi)^4} \int \ud^3 k\ud k^0\,
\frac{e^{\imath \vect{k}.\vect{z}} e^{- \imath k^0 z^0}}{k^{0^2} -\vect{k}^2}
\]
and note that if we perform the $k_0$ integral we see that the integrand
is singular at $k_0 = \pm \abs{\vect{k}}$.  We thus need
to choose on which contour to perform the integral.

\vspace{1in}

If we consider the retarded Green's function $G_{\text{ret}}$,
which we get by integrating along $\Gamma_1$, we see that
for $z^0 < 0$, $G(x,x') = 0$ as we can close the contour
in the upper half plane and apply Cauchy's theorem.  For
$z^0 > 0$ we have to close the contour in the lower half plane.  In doing
this we pick up two poles at $\pm \abs{\vect{k}}$ and can apply the residue
theorem.

The advanced Green's function $G_{\text{adv}}$ is obtained by
integrating along $\Gamma_2$.  In this case $G$ is only non-zero
for $z^0 > 0$.

The retarded Green's function agrees with intuitive
ideas of causality so we use that.  All we have to do now is
evaluate it.

\begin{align*}
G_{\text{ret}}(x,x') &= \theta(z^0) \frac{1}{(2 \pi)^4}
\int \ud^3 k \ud k^0\, \frac{e^{\imath \vect{k}.\vect{z}}
e^{- \imath k^0 z^0}}{k^{0^2} -\vect{k}^2 } \\
\intertext{We close the contour clockwise so we get
$\left\{- 2 \pi \imath \sum \text{residues}\right\}$ for the
$k^0$ integral, thus}
G_{\text{ret}}(x,x') &= - \frac{2 \pi \imath \theta(z^0)}{(2 \pi)^4}
\int \ud^3 k\, e^{\imath \vect{k}.\vect{z}}
\left( \frac{e^{- \imath \abs{\vect{k}} z^0}
- e^{\imath \abs{\vect{k}} z^0}}{2 \abs{\vect{k}}} \right). \\
\intertext{We convert this into spherical polars in $\vect{k}$-space:
$k_x = k \sin \theta \cos \phi$, $k_y = k \sin \theta \sin \phi$
and $k_z = k \cos \theta$ and so}
G_{\text{ret}}(x,x') &= - \frac{ \imath \theta(z^0)}{16 \pi^3}
\int k^2 \ud k\, \sin \theta \ud \theta\, \ud \phi\,
\frac{e^{\imath k z \cos \theta}}{k} \left(
e^{-\imath k z^0} - e^{\imath k z^0}
\right) \\
&= - \frac{\imath \theta(z^0)}{8 \pi^2}
\int k \ud k\, \sin \theta \ud \theta\,
e^{\imath k z \cos \theta} \left(e^{-\imath k z^0} - e^{\imath k z^0}
\right) \\
&= \frac{\theta(z^0)}{8 \pi^2 z}
\int_0^\infty \ud k\, \left[e^{\imath k z \cos \theta}
\right]_{\theta=0}^{\theta=\pi} \left(e^{-\imath k z^0} -
e^{\imath k z^0}\right). \\
\intertext{The integrand is even in $k$, so}
G_{\text{ret}}(x,x') &=
\frac{\theta(z^0)}{16 \pi^2 z}
\int_{-\infty}^\infty \ud k\, \left(
e^{-\imath k z} - e^{\imath k z}
\right) \left( e^{-\imath k z^0} - e^{\imath k z^0}\right).\\
\intertext{Recall that $\int_{-\infty}^\infty \ud k e^{\imath k x}
= 2 \pi \delta(x)$.  Thus the integral is a combination of four
delta functions, but the step function $\theta$ kills two of them off
and we get}
G_{\text{ret}}(x,x') &= - \frac{\theta(z^0)}{4 \pi z}
\delta(z - z^0).
\end{align*}

For the record, $G_{\text{adv}} = \frac{\theta(-z^0)}{4 \pi z}
\delta(z+z^0)$.  We can make our result for $G_{\text{ret}}$
look more covariant by recalling that
\[
\delta(f(x)) = \sum_i \frac{\delta(x-a_i)}{\abs{f'(a_i)}}
\]
where $f(a_i) = 0$.  Then as
$\delta^{(3)} (z^2) = \delta(\abs{\vect{z}}^2 - z^{0^2})$ ($z$
is a four-vector) we have
\[
\delta^{(3)}(z^2) = \frac{1}{2 \abs{\vect{z}}}
\left( \delta^{(3)}(\abs{\vect{z}} - z^0) + \delta^{(3)}(
\abs{\vect{z}} + z^0) \right)
\]
and as the step function removes one of these delta functions we get
\[
G_{\text{ret}}(x,x') = - \frac{1}{2 \pi} \theta(z^0)
\delta^{(3)}(z^2), \qquad z^\mu = x^\mu - x'{}^\mu.
\]

Now suppose we wish to evaluate $A^\mu(x)$ for some current
distribution $j^\mu$ as shown.  We get

\[
A^\mu(x) = \frac{\mu_0}{2 \pi} \int \ud^4 x'
j^\mu(x') \delta^{(3)}\!\left((x - x')^2\right) \theta(x^0 - x'{}^0).
\]

\vspace{1.5in}

This comes about because we chose $G_{\text{ret}}$.  The advanced
Green's function gives the reverse.  Thus $G_{\text{ret}}$ is
consistent with our ideas of causality.  Other choices of $G$
are not.  This choice goes beyond local physics.  It is presumably
solved by appealing to cosmology or quantum theory.

We also note that the only contributions to $A^\mu(x)$ come from
points $x'$ such that $(x-x')^2 = 0$ --- that is only when
$x$ and $x'$ can be joined by a light ray pointing towards the future of
$x'$.

\section{The field of a moving charge}

Suppose we have a moving charge, with (non-relativistically)
$\rho = e \delta^3(\vx - \vect{y}(t))$ and therefore
$\jc = e \diff{\vect{y}}{t} \delta^3(\vx - \vect{y}(t))$.  In the relativistic
case we replace $\vect{y}(t)$ with $y^\mu(\tau)$ and get
$j^\mu = e u^\mu \delta^3(x^i - y^i(t))$.  We can use a trick to make this
look more covariant,
\[
j^\mu = e \int \ud \tau\, u^\mu \delta^4(x^\nu - y^\nu(\tau)).
\]

Then
\[
A^\mu = \frac{e \mu_0}{2 \pi} \int \ud^4 x'\, \ud \tau \,
\delta\!\left((x-x')^2\right) \theta(x^0 - x'{}^0) \diff{y^\mu}{\tau}
\delta^4(x'{}^\nu - y^\nu(\tau)),
\]
which has the effect of integrating over the backward light cone.  This can
be evaluated by carrying out the $x'$ integral first.  Note that
\[
\delta\!\left((x-y)^2\right) = \frac{\delta\!\left(\abs{\vect{x} - \vect{y}}
- (x^0 - y^0)\right) + \delta\!\left(\abs{\vect{x} - \vect{y}}
+ (x^0 - y^0) \right)}{-2 (x-y)_\nu \diff{y^\nu}{\tau}}
\]

The second delta function does not contribute (because we are using the
retarded Green's function) and so
\begin{align*}
A^\mu &= \frac{e \mu_0}{2 \pi} \int \ud \tau\, \diff{y^\mu}{\tau}
\frac{\delta\!\left(\abs{\vect{x} - \vect{y}}
- (x^0 - y^0) \right)}{-2 (x-y)^\nu \diff{y_\nu}{\tau}} \\
&= \frac{e \mu_0}{4 \pi} \left. \frac{\diff{y^\mu}{\tau}}{\diff{y^\nu}{\tau}
(x-y)_\nu}\right|_{\tau = \tau_0}
\end{align*}
where $\tau_0$ is the value of $\tau$ on the world-line where the past light
cone of $x$ intersects the world line of the particle.  This is usually
referred to as evaluating at some instant of retarded time.
These are the Lienard-Wiechert potentials, and are painful to use in
arbitrary relativistic motion.

The result
\[
A^\mu = \frac{e \mu_0}{2 \pi} \int \ud \tau \,
\delta\!\left((x-y)^2\right) \theta(x^0 - y^0) \diff{y^\mu}{\tau}
\]
\emph{is} useful for relativistic motion.  To evaluate the fields we
need to calculate terms of the form
\begin{align*}
\partial^\nu A^\mu &= \frac{e \mu_0}{2 \pi}
\int \ud \tau\, \theta(x^0 - y^0) \diff{y^\mu}{\tau} \partial^\nu
\delta\!\left( (x-y)^2 \right) \\
&= \frac{e \mu_0}{2 \pi} \int \ud \tau\,
\theta(x^0 - y^0(\tau)) \diff{y^\mu}{\tau} \frac{x^\nu - y^\nu(\tau)}
{\diff{y^\rho}{\tau} (x-y)_\rho} \diff{}{\tau} \delta\!\left( (x-y)^2 \right)
\end{align*}

To evaluate this we integrate by parts.  We take $x^0 \neq y^0$ and thus
remove points on the world-line of the particle from consideration.  The
field is not well defined there.  Thus
\[
F^{\mu \nu} = - \frac{e \mu_0}{2 \pi}
\diff{}{\tau} \left[
\frac{(x-y)^\mu \diff{y^\nu}{\tau} - (x-y)^\nu \diff{y^\mu}{\tau}
}{\diff{y^\rho}{\tau} (x-y)_\rho}
\right]_{\tau = \tau_0}.
\]

We write $(x-y)^\mu = (+R, R\vect{n})$ where $R$ is the spatial distance
$\abs{\vect{x}-\vect{y}}$ and $\vect{n}$ is a unit vector.  The plus
sign on $R$ comes from the retarded Green's function.  We also need
the velocity $\vect{v} = (\gamma,\gamma \vect{v})$ where $\vect{v}
= \diff{\vect{y}}{t}$.  After evaluating this we get
\begin{align*}
\E &= \frac{e \mu_0}{2 \pi} \left[
\frac{\vect{n} - \vect{v}}{\gamma^2 (1-\vect{n}.\vect{v})^3 R^2}
+ \frac{\vect{n} \wedge \left\{(\vect{n} - \vect{v}) \wedge \dot{\vect{v}}
\right\}} {(1-\vect{n}.\vect{v})^3 R}
\right]_{\text{retarded time}} \\
\B &= \vect{n} \wedge \E.
\end{align*}

The first term in the expression for $\E$ is just the Coulomb field
(put $\vect{v} = 0$ to see this).  The second term only appears if
$\dot{v} \neq 0$ --- it depends on the acceleration.  Then
\[
\E \sim \frac{\text{acceleration}}{R} \quad \text{and} \quad
\B \sim \frac{\text{acceleration}}{R}
\]
and are perpendicular.  Thus the Poynting vector is $\vect{N} \sim
\frac{\text{acceleration}^2}{R^2}$.  Thus the energy flux out of
a large radius sphere $\sim \text{acceleration}^2$ --- accelerating particles
radiate energy.

For a non-relativistic particle it is somewhat easier.  We use the
Lienard-Wiechert potentials
\[
A^\mu = \frac{e \mu_0}{4 \pi} \left. \frac{\diff{y^\mu}{\tau}}
{\diff{y^\nu}{\tau}(x-y)_\nu} \right|_{\text{retarded time}}
\]
and put $x^\mu = (t,\vx)$, $y^\mu = (t',\vect{y})$ with $\vx - \vect{y}
= R \vect{n}$ where $\vect{n}$ is a unit vector.  For non-relativistic
motion $\diff{y^\mu}{\tau} = (1,\vect{v})$ and
\[
A^\mu = \frac{e \mu_0}{4 \pi} \left. \frac{(1,\vect{v})}{\abs{\vx - \vect{y}}}
\right|_{\text{at $t' = t - R$}}.
\]

It is straightforward to calculate $\E$ and $\B$ from $\B = \curl \vect{A}$
and $\E = -\grad A^0 - \dot{\vect{A}}$.  We first evaluate $\B$, and note that
we are only interested in the $\cO(R^{-1})$ terms --- to get the radiation
at infinity.  Thus
\[
\B \sim - \frac{e \mu_0}{4 \pi R} \vect{n} \wedge \dot{\vect{v}}
\quad \text{and} \quad
\E \sim - \frac{e \mu_0}{4 \pi R} \vect{n} \wedge \left[
\vect{n} \dot{\vect{v}} \right].
\]

The Poynting vector $\vect{N} = \frac{1}{\mu_0} \E \wedge \B$ evaluates as
$\vect{N} = \frac{e^2 \mu_0}{16 \pi^2 R^2} \left( \ddot{\vect{y}} \right)^2
\sin^2\! \theta\ \vect{n}$.

\vspace{1in}

The radiation is mainly perpendicular to the direction of the acceleration
and is axisymmetric about that axis.  $\dot{v}$ determines the time dependence
of the radiation and thus the frequency can be found by Fourier transforming
$\dot{v}$.

The power radiated (or the total flux of radiation) is
$\int \vect{N}.\vect{dS}$ over a sphere at infinity which we assume is at a
large distance from the particle (the celestial sphere).
Converting to polars we get
\[
\text{flux of radiation}
= \frac{e^2 \mu_0}{16 \pi^2} \left( \ddot{\vect{y}} \right)^2
\int \sin^3\! \theta\ \ud \theta \ud \phi = \frac{e^2 \mu_0}{6 \pi}
\left( \ddot{\vect{y}} \right)^2.
\]

This is Larmor's formula.

\subsection{Radiation reaction}

We consider a particle with mass $m$ and charge $e$ moving under an
external force $\vect{F}_{\text{ext}}$.  Assume (naively) Newton's Law
$m \ddot{\vect{y}} = \vect{F}_{\text{ext}}$ and dot this with
$\dot{\vect{y}}$ and integrate to get that the change in kinetic
energy equals the work done by the applied force.  But we know that
this is not true --- there are radiative losses at infinity.  We
therefore guess another force $\vect{F}_R$ and propose $m
\ddot{\vect{y}} = \vect{F}_{\text{ext}} + \vect{F}_R$.  Dotting this
with $\dot{\vect{y}}$ and integrating we see that $\vect{F}_R.\dot{\vect{y}}
$ is the radiative energy loss.  Using Larmor's formula and assuming
that there is no acceleration at the endpoints of the motion we get
$\vect{F}_R = \frac{e^2 \mu_0}{6 \pi} \dddot{\vect{y}}$ and derive the
Abraham-Lorentz equation
\[
m\left(\ddot{\vect{y}} - \frac{e^2 \mu_0}{6 \pi m} \dddot{\vect{y}} \right)
= \vect{F}_{\text{ext}}.
\]

This is very odd and leads to embarrassing difficulties.  To solve such
an equation three initial conditions are needed, position, velocity and
acceleration.  But if we take $\vect{F}_{\text{ext}} = \vect{0}$ we
see that the solutions are $A + B t + Ce^{\frac{t}{\tau}}$, where
$\tau$ is the timescale $\frac{e^2 \mu_0}{6 \pi \mu}$.  This exponential
runaway solution is presumed to be unphysical.

If $\vect{F}_{\text{ext}}$ is a delta function and we assume the
initial conditions $x = \dot{x} = 0$ we have to adjust $\ddot{x}$ to
suppress the runaway solution.  We see that the particle (if charged)
must start accelerating \emph{before} the force is applied.  This
acausal behaviour is called pre-acceleration and is governed by the
timescale $\tau$, which for an electron is approximately $6 \times
10^{-24} \mathrm{s}$, into the realm of quantum mechanical effects.

\section{Oscillating Fields}

Assume that $j^\mu(\vect{x},t) = j^\mu(\vect{x}) e^{\imath \omega t}$ with
$j^\mu$ non-zero only in some domain $D$.

\vspace{1.5in}

Using our result for $A^\mu$ we get
\[
A^\mu(\vect{x},t) = \frac{\mu_0}{4 \pi} e^{\imath \omega t}
\int \ud^3 x'\, j^\mu(\vect{x}') \frac{1}{\abs{\vect{x} - \vect{x}'}}
e^{-\imath \omega \abs{\vect{x}-\vect{x}'}}.
\]

If $R \gg d$ we can expand $\frac{1}{\abs{\vect{x}-\vect{x}'}}$ in the
usual way as
\begin{align*}
\frac{1}{\abs{\vect{x}-\vect{x}'}} &= \frac{1}{R}
\left( 1 - \frac{2 \vect{x}.\vect{x}'}{R^2}
+ \frac{\abs{\vect{x}'}^2}{R^2} \right)^{-\frac{1}{2}} \\
&= \frac{1}{R} \left( 1 + \frac{\vect{x}.\vect{x}'}{R^2} - \frac{1}{2}
\frac{\abs{\vect{x}'}^2}{R^2}
+ \frac{3}{2} \frac{\left(\vect{x}.\vect{x}' \right)^2}{R^4} \dots \right).
\end{align*}

The $\cO(R^{-2})$ and lower terms do not contribute to the radiation and will
be omitted.  We get
\[
A^\mu(\vect{x},t) = \frac{\mu_0}{4 \pi R} e^{\imath \omega t}
\int \ud^3 x' j^\mu(\vect{x}') e^{-\imath \omega \abs{\vect{x}
- \vect{x}'}}.
\]
We can perform a similar expansion on $e^{- \imath \omega \abs{\vect{x}
- \vect{x}'}}$ and finally get
\[
A^\mu(\vect{x},t) = \frac{\mu_0}{4 \pi R} e^{\imath \omega
\left(t - R\right)} \int \ud^3 x' j^\mu(\vect{x}') e^{\imath \omega
\frac{\vect{x}.\vect{x}'}{R}}
\]
provided $R \gg \lambda$, the wavelength.  Thus the expansion we
have derived is valid when $R \gg d, \lambda$.  This is called the radiation
zone.

Thus at large distances the system appears to be a source of spherical waves.
To proceed further we can expand out the phase factor in powers
of $\omega$.  We get
\[
A^\mu(\vect{x},t) = \frac{\mu_0}{4 \pi R} e^{\imath \omega \left( t - R
\right)} \int \ud^3 x'\, j^\mu(\vect{x}')
\left[
1 + \imath \omega \frac{\vect{x}.\vect{x}'}{R} -
\omega^2 \frac{\left( \vect{x}.\vect{x}'\right)^2}{2 R^2} + \dots
\right].
\]

In the radiation zone when $\omega d \gg 1$ these terms are successively
smaller.

Recall that $j^\mu = (\rho,\jc)$.  Then

\[
A^0(\vect{x},t) = \frac{\mu_0}{4 \pi R} e^{\imath \omega \left(
t - R \right)} \left[Q + \frac{\imath \omega}{R} \vect{x}.\vect{p} + \dots
\right]
\]
where $\vect{p}$ is the electric dipole moment of the system.  Note that
$Q = 0$ as the total charge cannot depend on time.  For the vector
potential,
\[
A^i(\vect{x},t) = \frac{\mu_0}{4 \pi R} e^{\imath \omega \left(
t - R\right)} \left[
\int \ud^3 x'\, j_i(\vect{x}') + \frac{\imath \omega}{R} x_j
\int \ud^3 x'\, x_j' j_i(\vect{x}') + \dots
\right]
\]
We can simplify this by noting (integrate by parts)
that
\[
\int \ud^3 x' \, j_i(\vect{x}') = - \int \ud^3 x'\, x_i' \partial_j
j_j(\vect{x}')
\]
and applying the continuity equation, which in this case is $
\imath \omega \rho + \dive \jc = 0$.  Thus
\[
\int \ud^3 x' \, j_i(\vect{x}') = \imath \omega \vect{p}
\]
and we get
\[
A^0 = \frac{\imath \omega \mu_0}{4 \pi R^2} e^{\imath \omega \left( t-R
\right)} \frac{\vect{p}.\vect{x}}{R} \quad \text{and} \quad
\vect{A} = \frac{\imath \omega \mu_0}{4 \pi R} e^{\imath \omega \left( t-R
\right)} \vect{p}.
\]

We can now calculate $\E$ and $\B$ as
\begin{align*}
\E &= \frac{\omega^2 \mu_0}{4 \pi R^3} e^{\imath \omega \left( t-R \right)}
\left(R^2 \vect{p} - (\vx . \vect{p}) \vx \right) \\
&= \frac{\omega^2 \mu_0}{4 \pi R^3} e^{\imath \omega \left( t-R \right)}
\vx \wedge \left( \vx \wedge \vect{p} \right) \quad \text{and}\\
\B &= \frac{\omega^2 \mu_0}{4 \pi R^2} e^{\imath \omega \left( t-R \right)}
\vx \wedge \vect{p}.
\end{align*}

The time averaged Poynting vector thus points radially outwards and has
magnitude $N = \frac{\mu_0 \omega^4 \abs{\vect{p}}^2}{32 \pi^2 R^2}
\sin^2 \theta$ and the average power radiated is therefore
$\frac{\mu_0 \omega^4 \abs{\vect{p}}^2}{12 \pi}$.

The scattered light has $\omega^4$ dependence times the spectrum of the
light.  Thus blue light is scattered preferentially to red and the sky
appears blue.  This also explains the red sun at sunset; since there is more
scattering when the angle of the sun is low and the blue light is scattered
more.

\chapter{Quantum mechanical effects}

\section{Minimal coupling}

Consider a particle with charge $e$, worldline $x^\mu(\tau)$ in an
electromagnetic field with potential $A^\mu$.  Recall we obtained an
action
\[
I = - \int \ud \tau\, \left( m \sqrt{- \dot{x}^2} - e A_\mu \dot{x}^\mu
\right) = \int \ud \tau\, \cL(x,\dot{x}), 
\]

where the minus sign is inessential; it just normalizes things nicely.  The
momentum $\pi^\mu$ conjugate to $x^\mu$ is $\pd{\cL}{\dot{x}^\mu}
= m \dot{x}^\mu + e A^\mu$, consisting of the mechanical momentum
and a modification due to the electromagnetic field.  The Hamiltonian
$H(x,\pi) = \pi^\mu \dot{x}_\mu - \cL = \frac{\left( \pi - e A\right)^2}{m}$.

The replacement of $p$ with $\pi = p + e A$ is usually termed ``minimal
coupling'' and corresponds to classical electrodynamics.

In quantum mechanics the momentum $p$ is replaced with a momentum operator
$\Hat{p}$, and we assume that the momentum operator for charged particles
is modified ``like the classical momentum'', that is
$p \mapsto \pi = \Hat{p} + e A \mapsto - \imath \hbar \nabla + e A$.

The Schr\"odinger equation for a static field $(0,\vect{A})$ is
$\frac{p^2}{2 m} \psi = E \psi$, which according to minimal coupling, and
turning $p$ into an operator, is $\frac{\left( - \imath \hbar \nabla
+ e \vect{A} \right)^2}{2 m} \psi = E \psi$.

Since gauge transformations are not supposed to have any physical effect
solutions of the Schr\"odinger equation in one gauge must be solutions in
another gauge.  If we start with the universal combination
$\left( -\imath \hbar \nabla + e \vect{A} \right) \psi$, on sending
$\vect{A} \mapsto \vect{A} + \nabla \Lambda$ the universal combination
becomes $\left( - \imath \hbar \nabla + e \vect{A}
+ e \nabla \Lambda \right) \psi'$.  This must be invariant
(up to a phase factor), and so if $\psi' = \psi e^{-\frac{\imath e \Lambda}
{\hbar}}$ we get $\left( -\imath \hbar \nabla  + e \vect{A} + e
\nabla \Lambda \right) \psi' = e^{-\frac{\imath e \Lambda}
{\hbar}} \left( - \imath \hbar \nabla - e \nabla \Lambda + e \vect{A}
+ e \nabla \Lambda \right) \psi$ and the universal combination is invariant
(up to a phase factor).  Phase should not be too disturbing; the matrix
element
\[
\int \ud^3 x\, \psi_1^\ast \Hat{O} \psi_2 \mapsto
\int \ud^3 x\, \psi_1^\ast e^{\frac{\imath e \Lambda}{\hbar}}
\Hat{O} \psi_2 e^{-\frac{\imath e \Lambda}{\hbar}}
\]

and under all normal circumstances the phase factors cancel; the matrix
element is invariant.

This minimal coupling means that the vector potential can give rise
to observable physical effects.  One which you may have met before is
the Aharonov - Bohm effect.

Consider the long, thin solenoid shown, with $\B \neq 0$ inside and
$\B = 0$ outside.  In classical mechanics, charged particles with be
unaffected since $\B = 0$ outside the solenoid.

In quantum mechanics; consider eigenstates of $\pi$; states $\psi$
with $\left( - \imath \hbar \nabla + e \vect{A} \right) \psi = \pi \psi$.
If the phases of the waves on the two paths differ then there will be
destructive interference.

Now suppose that a neutral particle has a wavefunction $\psi_0(x)$.  For
a charged particle the corresponding wavefunction is
$\psi(x) = \psi_0(x) \exp \left( - \frac{\imath e}{\hbar}
\int_{x_0}^x \vect{A}.\vect{\ud l} \right)$.  Thus the phase factor (the
difference in phase) between the two paths is
\begin{align*}
e^{-\frac{\imath e}{\hbar} \oint \vect{A}.\vect{\ud l}} &=
e^{-\frac{\imath e}{\hbar} \int \curl \vect{A}.\vect{\ud S}} \\
&= e^{-\frac{\imath e}{\hbar} \int \B.\vect{\ud S}} \\
&= e^{-\frac{\imath e}{\hbar} \left(\text{flux}\right)}.
\end{align*}

By appropriate choice of the flux $\Phi$ we can get as much or as little
interference as we want.  If $\frac{e}{\hbar} \Phi = \pi$ then there is
completely destructive interference; if $\frac{e}{\hbar} \Phi = 2 \pi$
then the interference is completely constructive and the solenoid is
undetectable.  In general if $\Phi = \frac{2 \pi n \hbar}{e}$ the
solenoid is unobservable.  This is an inherently quantum mechanical effect.

One might think that $\B = 0$ outside the solenoid implies that $\vect{A} = 0$
outside the solenoid.  This is true only if the region is simply connected
--- which it isn't.  We can make a gauge transformation to put $\vect{A} = 0$
at a point but because the region is not simply connected we cannot do this
everywhere.

This was experimentally verified in the 1960's.

\section{Conduction}

An ordinary conductor looks something like a regular lattice of atoms, with
the valence electrons forming an electron gas throughout the material.

\vspace{1in}

An applied $\E$ field moves the gas, but electrons collide with atoms
and stop.  Suppose they move with an average velocity $\vect{v}$.  Then
the current density is the charge on an electron $\times$ the number
density $\times \vect{v}$.  The mean free path only depends on the geometry,
so the current density is $\sigma \E$, with $\sigma$ the conductivity.

Superconductivity is very different.  It was first discovered by
Kammerlingh-Onnes in 1905; he noticed that when some metals are cooled
to $\approx 4 \K$ the electric conductivity became infinite.  Nowadays
superconductivity is observed in certain materials up to about liquid
nitrogen temperatures, $\approx 100 \K$.

The fundamental description of superconductivity is due to Bardeen,
Cooper and Schreiffer and is in detail beyond this course.  The result
is that the current is an inherently quantum mechanical effect in which
bound states of pairs of electrons behave as bosons rather than as fermions.
They have a charge $-2 e$ and an effective mass of $m$ (say).

We will examine the Landau-Ginzburg theory. Suppose the charge
carriers have a wavefunction $\chi = R e^{\imath \phi}$.  We can then
interpret the probability current as the flux of these particles.  We
can evaluate

\[
\jc_{\text{prob}} = \frac{\hbar}{2 \imath m} \left( \chi^\ast \nabla \chi
- \left( \nabla \chi \right)^\ast \chi \right)
= \frac{\hbar}{m} R^2 \nabla \phi.
\]

We interpret $R^2$ as a number density $n_s$ and so we guess an
electric current $\jc = \frac{q \hbar}{m} \nabla \phi$.  However this
is not gauge invariant and as the electric current must stay the same under
gauge transformations we fix up the equation to get the result
(which can be \emph{derived} from the BCS theory)
\[
\jc_s = \frac{q \hbar n_s}{m} \left( \nabla \phi - \frac{q}{\hbar} \vect{A}
\right).
\]

\subsection{Meissner effect}

Since $\dive \B = 0$ lines of $\B$ cannot end.  However if one
takes a material in a magnetic field and cools it to its superconducting
temperature one observes a change in the magnetic field.

\vspace{1in}

We are led to guess that $\B = 0$ inside a superconductor.  The above
expression for $\jc_s$ and the Maxwell equations give $\curl \B =
\frac{\mu_0 q \hbar n_s}{m} \left\{ \nabla \phi - \frac{q}{\hbar}
  \vect{A} \right\}$.  Taking the curl of this we get a differential
equation for $\B$:
\[
\nabla^2 \B = \frac{\mu_0 q^2 n_s}{m} \B.
\]

\vspace{1in}

In the region shown this simplifies to $\pd{^2 \B}{z^2} = \frac{\mu_0
  q^2 n_s}{m} \B$ and so we find that $\B = \B_0 \exp - \sqrt{
  \frac{\mu_0 q^2 n_s}{m}} z$, taking the negative root since the
energy must be bounded.

$\B$ decays exponentially away from the surface on a distance scale
$\sqrt{\frac{m}{n_s q^2 \mu_0}}$, which is of the order of atomic
size.  Thus in practice we have $\B = 0$ inside a superconductor and
this is a better definition of a superconductor than saying that it
has infinite conductivity.

$\vect{A}$ is not necessarily $0$, but in order to get a
superconducting current we must have $n_s \neq 0$.  Landau and
Ginzburg tried to construct an analog of the Schr\"odinger equation
which gave this result.

It is easier (as always) to start from an action:

\begin{align*}
I &= \int \ud^3 x\, \left[ \text{kinetic energy} - \text{potential energy}
\right] \\
&= \int \ud^3 x\, \left[- \frac{\hbar^2}{2m} \psi^\ast \nabla^2 \psi
- V \psi^\ast \psi \right] \\
&= \int \ud^3 x\, \left[\frac{1}{2m} \left( - \imath \hbar \nabla \psi
\right)^\ast \left( - \imath \hbar \nabla \psi\right) + V \psi^\ast \psi
 \right]\\
\intertext{incorporating the magnetic field via minimal coupling}
&= \int \ud^3 x\, \left[ \left( - \imath \hbar \nabla \psi
- q \vect{A} \psi \right)^\ast
\left( - \imath \hbar \nabla \psi - q \vect{A} \psi \right)
+ V \psi^\ast \psi \right].
\end{align*}

This is gauge invariant.

It cannot depend on where we are in the superconductor and so $V$
is constant.  We get the familiar Schr\"odinger equation which
has the obvious solution $\psi = 0$ and no other solution independent of
$\vect{x}$.

Landau and Ginzburg proposed the addition of a term
$\frac{1}{2} b \abs{\psi}^4$ to this action to get
\[
I = \int \ud^3 x\, \frac{\hbar^2}{4 m} \abs{\left(
\nabla - \frac{\imath q}{\hbar} \vect{A}
\right) \psi}^2 +
V \abs{\psi}^2 + \frac{1}{2} b \abs{\psi}^4.
\]

This action can be derived from BCS theory and gives a nonlinear
analog of the Schr\"odinger equation:
\[
- \frac{1}{4 m} \left( - \imath \hbar - q \vect{A}\right)^2  \psi
+ V \psi + b \abs{\psi}^2 \psi = 0.
\]

The current $\jc_s = \frac{q \hbar}{2 \imath m} \left(
\psi^\ast \nabla \psi - \psi \left( \nabla \psi \right)^\ast
\right) - \frac{2 q^2}{m} \vect{A} \abs{\psi}^2$.

We get a non-vanishing spatially independent solution of this
``Schr\"odinger equation'' when $V < 0$ and $b > 0$.  This
occurs when the temperature $T$ is less than some critical temperature
$T_c$; normal matter has $V > 0$.

BCS theory gives $b > 0$ and $V = V_0 \left( T - T_c \right)$.

\section{Superconducting flux quantisation}

Consider a ring of superconducting material as shown.

\vspace{1in}

In the material $\B = 0$ and $\jc = 0$.  Since $\jc \propto
\left( \nabla \psi - \frac{q}{\hbar} \vect{A}\right)$ we must have
$\vect{A} = \frac{\hbar}{q} \nabla \phi$ inside the ring.

The magnetic flux through the loop is

\[
\int_{\text{shaded surface}} \B.\vect{\ud S}
= \oint_{\text{boundary loop}} \vect{A}.\vect{\ud l}.
\]

Evaluating this inside the superconductor we get
$\frac{\hbar}{q} \left[ \phi \right]$.  As the wavefunction must
be single valued this must be $n \frac{2 \pi \hbar}{q}$ and since
the charge carriers are electron pairs then the flux is quantised in
units of $\frac{\pi \hbar}{\abs{e}}$.

If we make a current $I$ flow on the surface of the superconductor then
as the flux through the loop is the inductance times the current, and so
the flux is quantised we see that the current is quantised.

\section{Magnetic monopoles}

Suppose that a $\B$ field $\frac{\mu_0}{4 \pi} P
\frac{\Hat{\vect{r}}}{r^2}$ is possible, by analogy with the Coulomb
field in electrostatics.

Using Gauss' Law we have
\[
P = \frac{1}{\mu_0} \int_{\text{closed surface}}
\B.\vect{\ud S} = \frac{1}{\mu_0} \int \dive \B\, \ud V.
\]

Thus if $\dive \B = 0$ everywhere then $P = 0$ and magnetic charges
cannot arise.  Thus Maxwell's equations must be modified in order to
get this field.

A suitable vector potential $\vect{A}$ is $A_\phi = \frac{\mu_0 P}{4
  \pi} \left( 1 + \cos \theta \right)$ (in spherical polars).  We have
$\abs{A} = \frac{\mu_0}{4 \pi} \frac{\left(1 + \cos \theta\right)}{r
  \sin \theta}$.  There is a difficulty at $\theta = 0$ for all $r$.

This singularity on the North axis is called the Dirac string.  It can
be moved about by gauge transformations; if we have $\vect{A} \mapsto
\vect{A} + \nabla \frac{- \mu_0 P}{2 \pi} \phi$ we can put the Dirac
string onto the South axis.

Since (by axiom) the observable physics should not depend on the gauge
used the string singularity should be unobservable.

We showed earlier that the phase difference between two paths going in
front of / behind the string is $\frac{e}{\hbar} \oint
\vect{A}.\vect{\ud l}$ and this must be an integer multiple of $2
\pi$.  Evaluating the integral gives the Dirac quantisation condition
$P = n \frac{2 \pi \hbar}{\mu_0 e}$.

\chapter{Born-Infeld Theory}

\marginpar{%
This material is starred and was included as a fill-in lecture.}

Recall that Maxwell's theory is (in the absence of currents) governed
by an action
\[
I = \frac{1}{\mu_0} \int \ud^4 x - \frac{1}{4} F_{\mu \nu} F^{\mu \nu},
\]

giving the two Maxwell equations $\partial_\mu F_{\rho \sigma} +
\partial_\rho F_{\sigma \mu} + \partial_\sigma F_{\mu \rho} = 0$ and
$\partial_\mu F^{\mu \nu} = 0$.  There is a hidden duality symmetry
under $F_{\mu \nu} \mapsto \frac{1}{2} \varepsilon_{\mu \nu \rho
  \sigma} F^{\rho \sigma}$ of both the action and the equations of
motion.

Recall also that electric charges have a radial component
$E_r = \frac{Q}{4 \pi \epsilon_0} \frac{1}{r^2}$ and that the energy
density in the electric field is $\frac{1}{2} \epsilon_0 \abs{\E}^2$.
We can see that the energy density blows up at the origin and also that
the total energy in the electric field is infinite.

We also propose a similar magnetic monopole field $B_r = \frac{P \mu_0}{4
\pi} \frac{1}{r^2}$; the energy in this magnetic field is also infinite.

The Born-Infeld theory emerges from string theory.  It depends on
a parameter $b$ with the dimensions of length.  We take a new action;
\[
\frac{1}{\mu_0 b^2} \int \ud^4 x\,\left\{ 1 - \sqrt{\abs{\det{\eta_{\mu \nu}
+ b F_{\mu\nu}}}} \right\}
\]
and we suppose that $F_{\mu \nu} = \partial_\mu A_\nu - \partial_\nu A_\mu$.

Since we have that, up to a Lorentz transform,
\[
F_{\mu \nu} = \Tilde{F}_{\mu \nu} = \begin{pmatrix} 0 & \lambda_1 & 0 & 0 \\
-\lambda_1 & 0 & 0 & 0 \\
0 & 0 & 0 & \lambda_2 \\
0 & 0 & -\lambda_2 & 0
\end{pmatrix}
\]
we can see that $\det \eta_{\mu \nu} + b \Tilde{F}_{\mu \nu} = \det
\eta_{\mu \nu} + b F_{\mu \nu}$ and so we can evaluate the action as
\[
\frac{1}{\mu_0 b^2} \int \ud^4 x  \left\{
1 - \sqrt{1 - \frac{1}{2} b^2 F_{\mu \nu} F^{\mu \nu} - \frac{1}{16}
b^4 \left( \varepsilon^{\mu \nu \rho \sigma} F_{\mu \nu} F_{\rho \sigma}
\right)^2 } \right\}.
\]

The limit $b \to 0$ (clearly) gives the Maxwell action.  Since we
are assuming that $F_{\mu \nu}$ is derived from a potential we still have
the equation $\partial_\mu F_{\nu \rho} + \partial_\nu F_{\rho \mu}
+ \partial_\rho F_{\mu \nu} = 0$ and the other equation is
$\partial_\mu \cG^{\mu \nu} = 0$, the difference being that
the equation for $\cG^{\mu \nu}$ is a horrible mess:
\[
\cG^{\mu \nu} = \frac{F^{\mu \nu} - \tfrac{b^2}{4} F^{\lambda \tau}
\varepsilon_{\mu \nu \lambda \tau} F_{\alpha \beta} F_{\gamma \delta}
\varepsilon^{\alpha \beta \gamma \delta}}{
\sqrt{1 - \frac{b^2}{2} F_{\xi \zeta} F^{\xi \zeta} - \frac{b^4}{16}
\left( \varepsilon_{\xi \zeta \chi \varpi} F^{\xi \zeta} F^{\chi
\varpi} \right)^2}}
\]

(you have no idea how difficult it was to find that many different Greek
letters).

We obtain
\[
T_{\mu \nu} = \frac{1}{\mu_0} \left\{ \cG_\mu{}^\lambda F_{\nu \lambda}
 + \frac{1}{4} \eta_{\mu \nu} \cL \right\}
\]
where $\cL$ is the Lagrangian; $\cL = \frac{1}{b^2} \left\{
1- \sqrt{\abs{\det \eta_{\mu \nu} + b F_{\mu \nu}}} \right\}$.

There is a symmetry in these equations under $F_{\mu \nu}
\mapsto \frac{1}{2} \varepsilon_{\mu \nu \rho \sigma} \cG^{\rho \sigma}$
and $\cG_{\mu \nu} \mapsto - \frac{1}{2} \varepsilon_{\mu \nu \rho \sigma}
F^{\rho \sigma}$:
the Lagrangian and equations of motions are invariant.  This has the effect
of swapping $\E$ and $\B$.

The analog of the electric field of a point charge is an $\E$ field
which is purely radial, defined by $A_0 = \phi$ and
$E_r = - \nabla_r \phi$.  In this case the action reduces to
\[
\frac{1}{\mu_0 b^2} \int r^2 \ud r \sin \theta \ud \theta \ud \phi
\left[ 1 - \sqrt{1 - b^2 \phi_r^2} \right]
\]
and variation of this yields
\[
\frac{r^2 \phi_r}{\sqrt{1 - b^2 \phi_r^2}} = \text{const} = a.
\]

Solving this for $\phi_r = -E_r$ yields
\[
E_r = - \frac{a}{\sqrt{r^4 + a^2 b^2}}
\]
and so as $r \to \infty$, $E_r \sim -\frac{a}{r^2}$.  Thus if we wish
to reproduce the Maxwell field for large distances $a = -\frac{Q}{4
  \pi \epsilon_0}$.  Thus as $r \to 0$ we see that $E_r \to \frac{1}{b}$.

The energy density in the electric field is
\[
\frac{1}{\mu_0 b^2} \left\{\frac{1}{\sqrt{1 - b^2 \E^2}} - 1\right\}
= \frac{1}{\mu_0 b^2} \left\{\sqrt{1 + \frac{b^2 Q^2}{16 \pi^2
\epsilon_0 r^4}} - 1 \right\}
\]

which is singular (but integrably so) at $r=0$.  Performing the integral
to find the total energy in the electric field we obtain
\[
\frac{4 \pi}{3 \sqrt{b}} \frac{1}{\Gamma(\tfrac{3}{4})^2} \left( \frac{Q}{4}
\right)^{\frac{3}{2}} \frac{1}{\epsilon_0^{\frac{1}{2}}},
\]
which is noticably finite.

This theory also has magnetic monopoles; an easy way is to see that the
theory is invariant under swapping $\E$ and $\B$.  The energy in
a magnetic monopole field is
\[
\frac{4 \pi}{3 \sqrt{b}} \frac{1}{\Gamma(\tfrac{3}{4})^2}
\left( \frac{P}{4}\right)^{\frac{3}{2}} \sqrt{\mu_0}
\]
if $\B_r \sim \frac{\mu_0 P}{4 \pi r^2}$ as $r \to \infty$.

In fact $B_r = \frac{\mu_0 P}{4 \pi r^2}$ which although it looks singular
is perfectly reasonable.

\backmatter

\begin{thebibliography}{9}
\bibitem{Jackson} J.D.~Jackson, \emph{Classical Electrodynamics},
  Second ed., Wiley, 1975.
  
  {\sffamily \small This is probably the best book for the course.  It
    is very complete and reasonably easy to read.  It has two major
    flaws; its price and its use of CGS units.  Every college library
    should have at least one copy. }

\bibitem{Feynman} Feynman, Leighton, Sands, \emph{The Feynman Lectures
    on Physics Vol. 2}, Addison-Wesley, 1964.
  
  {\sffamily \small The Feynman lectures are always good reading.
    However Feynman II, the book on electromagnetism, is not really
    suitable as a textbook for this course.  It is not complete
    enough although it is good for introductory reading and revision
    of earlier work. It's also fun... }

\bibitem{LandL} Landau and Lifschitz, \emph{The Classical Theory of
    Fields}, Fourth ed., Butterworth-Heinemann, 1975.
  
  {\sffamily \small This is a good reference but isn't so good to
    learn by.  It also uses CGS units.}
\end{thebibliography}

\begin{center}
\fbox{\parbox{3.5in}{
The sign conventions for the tensor $F_{\mu \nu}$ are variable and it
is essential to check which convention any given book is using.}}
\end{center}

There appears to be a gap in the market at about the level of this
course.  Most books on electromagnetism seem to be written either for
a first course or for a postgraduate course.  If you find a good one I
haven't mentioned please send me a \emph{brief} review and I will
append it to this bibliography if I think it is suitable.

\end{document}

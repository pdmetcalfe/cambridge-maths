\documentclass{notes}

\newcommand{\matd}[1]{\DDt{#1}}
\newcommand{\bso}{\boldsymbol\omega}
\newcommand{\grad}{\nabla}
\newcommand{\divr}{\nabla\cdot}
\newcommand{\curl}{\nabla\wedge}
\newcommand{\lapl}{\nabla^2}
\newcommand{\dgrad}[2]{\left(#1\cdot\nabla\right)#2}

\begin{document}

\frontmatter
\title{Fluid Dynamics}
\lecturer{Dr.~J.R.~Lister}
\maintainer{Paul Metcalfe}
\date{Mich\ae lmas 1996} \maketitle

\thispagestyle{empty}

\noindent\verb$Revision: 1.8 $\hfill\\
\noindent\verb$Date: 2001/11/01 14:44:11 $\hfill\

\vspace{1.5in}

The following people have maintained these notes.

\begin{center}
\begin{tabular}{ r  l}
-- date & Paul Metcalfe
\end{tabular}
\end{center}

\tableofcontents

\chapter{Introduction}

These notes are based on the course ``Fluid Dynamics'' given by
Dr.~J.R.~Lister in Cambridge in the Mich\ae lmas Term 1996.  These
typeset notes are totally unconnected with Dr.~Lister.

\alsoavailable
\archimcopyright

\mainmatter

\chapter{Kinematics}

\section{Continuum Fields}

Everyday experience suggests that at a macroscopic scale, liquids and
gases look like smooth continua with density $\rho(\vx,t)$, velocity
$\vu(\vx,t)$ and pressure $p(\vx,t)$ fields.

Since fluids are made of molecules this is of course only an
approximate description.  On large lengthscales we can define these
fields by averaging over a volume $V$ smaller than the scale of
interest but large enough to contain many molecules.  The effect of
this averaging is to exchange an enormous number of ODEs that describe
the motion of each molecule for a few PDEs that describe the averaged
fields.

This \emph{continuum approximation} is not always appropriate.  For
instance the velocity structure about a spacecraft during re-entry has
a lengthscale comparable with the molecular mean free path.
Similarly, blood flow in capillaries must take the red blood cells
into account.

\section{Flow Visualization}

There are many experimental techniques for obtaining a description of
the velocity field $\vu(\vx,t)$.  Three simple visualisation techniques
give rise to the ideas of \emph{streamlines}, \emph{pathlines} and
\emph{streaklines}.  We will illustrate these ideas by application to
the simple two-dimensional example $\vu(\vx,t) = (t,y)$.

\emph{Streamlines} are curves that are everywhere parallel to the
instantaneous flow.  They are visualised experimentally by the
short-time exposure of many brightly-lit particles --- the streamlines
are obtained by joining the resulting short segments in a manner
analogous to obtaining magnetic fields from iron filings.

Mathematically, a streamline is a curve $\vx(s;\vx_0,t)$ at a given
fixed time $t$ with $s$ varying along the curve and passing through a
given point $\vx_0$ that satisfies

\[
\pd{\vx}{s} = \vu(\vx,t) \qquad \vx(0;x_0,t) = \vx_0.
\]

For our example we have

\[
x(s;\vx_0,t) = x_0 + t s \qquad y(s;\vx_0,t) = y_0 e^s.
\]

This gives the curve
\[
\frac{x - x_0}{t} = \log \frac{y}{y_0}.
\]

\emph{Pathlines} are particle paths: paths traversed by particles
moving with the flow.  They are visualised experimentally by the
long-time exposure of a few brightly-lit particles.

Mathematically, a pathline is a curve $\vx(t;\vx_0,t_0)$ corresponding
to a particle released from $\vx = \vx_0$ at $t = t_0$.  The
differential equation is

\[
\pd{\vx}{t} = \vu(\vx,t) \qquad \vx(t_0;x_0,t_0) = \vx_0.
\]

Our example gives
\[
x(t;\vx_0,t_0) = x_0 + \frac{t^2-t_0^2}{2} \qquad y(t;\vx_0,t_0) = y_0
e^{t - t_0}.
\]

For a particle released at $t_0 = 0$ this gives a curve
\[
y = y_0 e^{\sqrt{2(x-x_0)}}.
\]

\emph{Streaklines} give the position at some fixed time of dye
released over a range of previous times from a fixed source (e.g. an
oil spill).

Mathematically, a streakline is a curve $\vx(t_0;\vx,t)$ with $t_0$
varying along the curve and a fixed observation time $t$.  To obtain
it, we still solve

\[
\pd{\vx}{t} = \vu(\vx,t) \qquad \vx(t_0;x_0,t_0) = \vx_0,
\]

but then fix $t$ instead of $t_0$.  Suppose we observe our flow at $t
= 0$ --- we can use our previous solution to get a streakline
\[
y = y_0 e^{- t_0} = y_0 e^{-\sqrt{2(x_0 - x)}}.
\]

Note that for this unsteady flow we get different results from each
method of visualisation.  The different methods give the same result
if the flow is steady.

\section{Material Derivative}

This is a rate of change ``moving with the fluid''.  For any quantity
$F$, the rate of change in that quantity seen by an observer moving
with the fluid is the material or Lagrangian derivative $\matd{F}$.

\begin{align*}
  \delta F &=
  \frac{F(\vx+\vect{\delta x},t+\delta t) - F(\vx,t)}{\delta t} \\
  &= \dgrad{\vect{\delta x}}{F} + \delta t \pd{F}{t} + \text{ smaller
    terms.}
\end{align*}

Hence
\[
\matd{F} = \dgrad{\vu}{F} + \pd{F}{t}.
\]

\section{Conservation of Mass}

Consider an arbitrary (at least smooth -- this is applied maths)
volume $V$, fixed in space with bounding surface $A$ and outward
normal $\vn$.  The mass inside $V$ is
\[
M = \int_V \rho \,\ud V,
\]
and the mass changes due to the flow over the boundary, so
\[
\pd{M}{t} = -\int_A \rho \vu\cdot\vn \,\ud A.
\]

Application of the divergence theorem gives

\[
\int_V \pd{\rho}{t} \,\ud V + \int_V \divr{(\rho \vu)} \,\ud V = 0.
\]

Since $V$ is arbitrarily small,

\[
\pd{\rho}{t} + \divr{(\rho \vu)} = 0,
\]
or rewritten using the material derivative
\[
\matd{\rho} + \rho \divr{\vu} = 0.
\]

\section{Kinematic Boundary Condition}

This is an expression of mass conservation at a boundary.  If the
velocity of the boundary is $\vu^A$, the condition for no mass flux is

\[
\rho \left( \vu(\vx,t) - \vu^A(\vx,t) \right)\cdot\vn \delta A \delta t =
0,
\]
which gives that $\vu\cdot\vn = \vu^A\cdot\vn$.  For a fixed surface,
$\vu^A = 0$, so the surface is a streamline.

\section{Incompressible Fluids}

For this course, we restrict ourselves to fluids with $\rho = const$.
Mass conservation reduces to $\divr{\vu} = 0$.  Such a velocity field
$\vu$ is said to be solenoidal.

\section{Streamfunctions}

This gives a representation of the flow satisfying $\divr{\vu}=0$
automatically.  For example, in 2D Cartesians, any velocity field
$\vu=(u,v,0)$ is solenoidal if there exists $\psi(x,y,t)$ such that $u
= \pd{\psi}{y}$ and $v = -\pd{\psi}{x}$.

In 2D polars, we want $\psi$ such that $u_r =
\frac{1}{r}\pd{\psi}{\theta}$ and $u_{\theta} = - \pd{\psi}{r}$.

In axisymmetric cylindrical polars, we want $\Psi$ such that $u_z =
\frac{1}{r} \pd{\Psi}{r}$ and $u_r = -\frac{1}{r}\pd{\Psi}{z}$.
$\Psi$ is called a Stokes streamfunction.

In axisymmetric spherical polars, we want $\Psi$ such that $u_r =
\frac{1}{r^2 \sin \theta}\pd{\Psi}{\theta}$ and $u_{\theta} =
\frac{-1}{r \sin \theta} \pd{\Psi}{r}$.

\chapter{Dynamics}

\section{Surface and volume forces}

Two types of force are considered to act on a fluid: those
proportional to volume (e.g. gravity) and those proportional to area
(e.g. pressure).  This is a simplification appropriate to the
continuum level description --- e.g. surface forces in a gas are the
average result of many molecules transferring momentum by collision
with other molecules over the very short distance of the mean free
path.

\subsection*{Volume forces}

We denote the force on a small volume element $\delta V$ by
$\vF^V(\vx,t) \delta V$.  The volume force is often conservative, with a
potential energy per unit volume $\chi$, so that $\vF^V = -
\grad\chi$ (or potential energy per unit mass $\Phi$, so that $\vF^V
= - \rho \grad\chi$.

The most common case is
\[
\vF^V(\vx,t) \delta V = \rho \vect{g} \delta V.
\]

\subsection*{Surface forces}

We denote the force on a small surface element $\vn \delta A$ by
$\vF^A(\vx,t,\vn) \delta A$, which depends on the orientation $\vn$ of the
surface element.  A full description of surface forces includes the
effects of friction of layers of water sliding over each other or over
rigid boundaries (viscosity).

Viscous effects are important when the \emph{Reynolds number}
\[
\frac{U L}{\nu} \le 1,
\]
where $U$ is a typical velocity, $L$ a typical length and $\nu$ is the
dynamic viscosity, which is a property of the fluid.

In many cases fluids act as nearly frictionless and in this course we
neglect frictional forces completely.  For a treatment of viscous
fluids see the Fluid Dynamics 2 course in Part IIB.

For inviscid (frictionless) fluids the surface force is simply
perpendicular to the surface with a magnitude independent of
orientation:

\[
\vF^A(\vx,t) \delta A = - p(\vx,t) \vn \delta A,
\]

where $p$ is the pressure.  The minus sign is so that pressure is
positive.

\section{Momentum Equation}

Consider an arbitrary (at least smooth -- this is applied maths)
volume $V$, fixed in space with bounding surface $A$ and outward
normal $\vn$.  The momentum inside $V$ is
\[
\int_V \rho \vu \,\ud V,
\]
and the momentum changes due to the flow over the boundary, surface
forces and volume forces, so

\begin{equation}\label{eq:momint}
\frac{\ud}{\ud t} \int_V \rho \vu \,\ud V = -\int_A \rho \vu (\vu\cdot\vn)
\,\ud A + \int_A -p \vn \,\ud A + \int_V \vF^V \,\ud V,
\end{equation}
which is the momentum integral equation.  Written in component form
\[
\frac{\ud}{\ud t} \int_V \rho u_i \,\ud V = -\int_A \rho u_iu_jn_j
\,\ud A + \int_A -p n_i \,\ud A + \int_V F_i^V \,\ud V
\]
$\rho u_iu_j$ is called the momentum flux tensor.

$V$ is fixed, so LHS is $\int_V \pd{\rho u_i}{t} \,\ud V$, and using
the divergence theorem on the RHS, then letting $V$ be arbitrarily
small, we achieve the Euler momentum equation
\begin{equation}\label{eq:eumom}
\rho \left( \pd{\vu}{t} + \dgrad{\vu}{\vu} \right) =-\grad p + \vF^V.
\end{equation}

The associated dynamic boundary condition is that given forces are
applied at the boundary (i.e. $p$ is given).

\subsection{Applications of integral form}

\subsubsection*{Uncoiling of hosepipes}

\vspace{1.5in}

Assume a steady uniform flow $U$ through a pipe of constant
cross-section $A$.   Neglect gravity.  Now \eqref{eq:momint} becomes
\[
\int_{\text{walls}} + \int_{\text{ends}} \left( \rho \vu (\vu \cdot \vn)
  + p \vn \right)\, \ud A = 0.
\]

The integral over the walls is
\[
\int_{\text{walls}} p \vn\,\ud A = \text{force on pipe,}
\]

since $\vu \cdot \vn = 0$ on the walls.  The integral over the ends is
\[
( \rho U^2 + p) A (\vect{2} - \vect{1})
\]

and so the force on the pipe is $\vF = ( \rho U^2 + p) A (\vect{1} - \vect{2})$.

\subsubsection*{Pressure change at abrupt junction}

\vspace{3in}

Apply a momentum balance to the sketched shape.  Neglect gravity, and
also neglect the time derivative, which is zero on average.

The momentum integral equation \eqref{eq:momint} becomes

\[
\int \left( \rho \vu (\vu \cdot\vn) + p \vn\right)\,\ud A = 0.
\]

The horizontal component gives

\[
\rho u_1^2 A_1 + p_1 A_1 = \rho u_2^2 A_2 + p_2 A_2, 
\]
and mass conservation gives $u_1 A_1 = u_2 A_2$.  Then we see that
\[
p_2 -p_1 = \rho u_1^2 \frac{A_1}{A_2} \left( 1 - \frac{A_1}{A_2} \right) > 0.
\]

\section{Bernoulli's Theorem}

For a steady flow ($\pd{\vu}{t}=0$) with potential forces ($\vF_V =
-\grad\chi$),
\[
\rho \dgrad{\vu}{\vu} = -\grad(p+\chi),
\]
which can be written
\begin{equation}\label{eq:bern1}
\rho \left( \frac{1}{2}\grad u^2 - \vu \wedge \left( \curl\vu \right)
\right) = -\grad (p+\chi).
\end{equation}

We define the \emph{vorticity} $\bso = \curl{\vu}$ and then let $H =
\frac{1}{2}\rho u^2 + p + \chi$.  Then $\grad H = \rho \vu \wedge
\bso$.  Now $\vu \cdot \grad{H} = 0$, so $H$ is constant on streamlines.
This is Bernoulli's theorem.

Note also that $\bso \cdot \grad{H} = 0$ and that $H$ is constant on
vortex lines.

The constancy of $H$ means that $p$ is low at high speeds.

\subsection{Application}

Consider a water jet hitting an inclined plane.

\vspace{1.5in}

Neglect gravity, so that on the surface streamline $p=p_a$ the speed
is constant.  Let this speed be $U$.

Now apply the momentum integral equation \eqref{eq:momint} to get
\[
\int_A \rho \vu \vu \cdot \vn + (p - p_a) \cdot \vn\,\ud A = 0. 
\]

Now $\vu \cdot \vn = 0$ except at the ends.  Mass conservation gives
\[
\rho a U = \rho a_1 U + \rho a_2 U.
\]

Now balance the momentum parallel to the wall to get
\[
\rho a U^2 \cos \beta = \rho a_2 U^2 - \rho a_1 U^2.
\]

Thus
\[
a_1 = \frac{1 + \cos \beta}{2} a \qquad \text{and} \qquad
a_2 = \frac{1 - \cos \beta}{2} a.
\]

Balancing the momentum perpendicular to the wall we get
$F = \rho a U^2 \sin \beta$.

\section{Vorticity and Circulation}

\subsection{Vorticity Equation}

Start with the Euler momentum equation \eqref{eq:eumom} with potential forces
\[
\rho \left( \pd{\vu}{t} + \dgrad{\vu}{\vu} \right) =-\grad{(p + \chi)}
\]
and take its curl to obtain
\[
\pd{\omega}{t} = \dgrad{\bso}{\vu} - \dgrad{\vu}{\bso} + \vu \divr{\bso}
- \bso\divr{\vu}.
\]
Now $\divr{\vu} = 0$ and $\divr{\bso} = 0$, so we obtain
\[
\pd{\bso}{t} = \dgrad{\bso}{\vu} - \dgrad{\vu}{\bso}
\] or
\begin{equation}\label{eq:vorteq}
\matd{\bso} = \dgrad{\bso}{\vu}.
\end{equation}

\subsection{Interpretation of vorticity}

Consider a material line element (ie a line element moving with the
fluid).  Then in a time $\delta t$, $\delta \vect{l} \rightarrow
\delta \vect{l} + \dgrad{\delta \vect{l}}{\vu}\delta t$, which gives
that $\pd{\delta \vect{l}}{t} = \dgrad{\delta \vect{l}}{\vu}$.  Hence
the tensor $\pd{u_i}{x_j}$ determines the local rate of deformation of
line elements.

\begin{align*}
  \pd{u_i}{x_j} &= \frac{1}{2} \left( \pd{u_i}{x_j} + \pd{x_j}{u_i}
  \right)
  + \frac{1}{2} \left( \pd{u_i}{x_j} - \pd{x_j}{u_i} \right) \\
  &=e_{ij} + \frac{1}{2} \epsilon_{jik}\omega_k.
\end{align*}

The local motion due to $e_{ij}$ is called the strain.  The motion due
to the second term $\frac{1}{2}\epsilon_{jik}\omega_k \delta l_j =
\frac{1}{2} \left( \bso \wedge \delta \vect{l} \right)$ is rotation
with angular velocity $\frac{1}{2} \bso$.

\subsection{Ballerina effect and vortex line stretching}

The vorticity equation \eqref{eq:vorteq} can be interpreted as saying
that vorticity changes just like the rotation and stretching of
material line elements.  This is just the conservation of angular
momentum.

Consider a rotating fluid cylinder, initially with angular velocity
$\omega_1$, radius $a_1$ and length $l_1$.  Conservation of mass gives
$a_1^2 l_1 = a_2^2 l_2$ and conservation of angular momentum gives
$a_1^4 l_1 \omega_1 = a_2^4 l_2 \omega_2$.  These combine to give
\[
\frac{\omega_1}{\omega_2} = \frac{l_1}{l_2},
\]
which says that vorticity increases as the fluid is stretched.  This
explains the bathtub vortex.

\subsection{Kelvin's Circulation Theorem}

Assume $\rho$ constant and $\vF^V = -\grad{\chi}$.  Define the
circulation $C(t)$ around a closed material curve $\Gamma(t)$ by
\[
C(t) = \oint_{\Gamma(t)} \vu(\vx,t)\cdot\vect{dl}.
\]

Then
\[
\pd{C(t)}{t} = \oint_{\Gamma(t)} \matd{\vu}\cdot\vect{dl} +
\vu\cdot\frac{\ud}{\ud t}\, \vect{dl}
\]
since $\Gamma$ moves with the fluid.  But from the momentum equation
$\matd{\vu} = -\grad{\frac{p+\chi}{\rho}}$ and $\frac{\ud}{\ud
  t}\vect{dl}= \dgrad{\vect{dl}}{\vu}$.  Hence

\begin{align*}
  \pd{C(t)}{t} &= \oint_{\Gamma(t)}
  \left(\grad \left(\frac{1}{2}u^2 -  \frac{p+\chi}{\rho} \right) \right)
  \cdot\vect{dl} \\
  &=0 \qquad \text{since $\Gamma$ is closed}
\end{align*}

So, for an inviscid fluid of constant density with potential forces,
the circulation around a closed material curve is constant.

\subsection{Irrotational flow remains so}

A flow with $\bso = 0$ is said to be irrotational.  If $\bso=0$
everywhere at $t=0$, then the vorticity equation \eqref{eq:vorteq}
becomes $\matd{\bso} = 0$, implying that $\bso=0$ for all times $t \ge
0$.

This isn't quite true, vorticity can leave a stagnation point,
especially at sharp trailing edges.

\chapter{Irrotational Flows}

You will want to find a table of vector differential operators in
various co-ordinate systems.  There is one in the back of
Acheson.

\section{Velocity potential}

The vorticity equation tells us that an initially irrotational flow
remains so for all time.  Since $\curl\vu = 0$ for all time there
exists a velocity potential $\phi(\vx,t)$ such that $\vu = \grad\phi$.
Note both the $+$ sign and that any function $f(t)$ can be added to
$\phi$.  Given $\vu$, we can find $\phi$ from

\[
\phi = \int_{\vx_0}^\vx \vu(\vx,t)\cdot\vect{dl}.
\]

The result is independent of path since $\curl\vu = 0$.  Note that
$\phi$ can be multivalued in 2D if there are holes with
\[
\oint_{\text{hole}} \vu \cdot \vect{dl} \neq 0.
\]

Mass conservation for an incompressible fluid reduces to $\cdot\phi =
0$, and so if $\vu = \grad\phi$ we have to solve $\lapl\phi = 0$.%
\footnote{Hence irrotational flows are sometimes called potential or
  harmonic flows.}

The kinematic boundary condition $\vu^A \cdot \vn = \vu \cdot \vn$ is
\[
\vu^A\cdot\vn = \vn\cdot\grad\phi \equiv \pd{\phi}{n}.
\]

Thus solving the Euler momentum equation \eqref{eq:eumom} reduces to
solving the more familiar Laplace's equation with Neumann boundary
conditions.  The flow is only non-zero because of the applied boundary
conditions on $\pd{\phi}{n}$ (on bodies, surfaces and at infinity).

\section{Some simple solutions}

For simple geometries it is often possible to write down solutions of
$\lapl\phi = 0$ as a sum of separable solutions in suitable
co-ordinate systems.  See the Methods course for details.

\subsection*{Cartesians}

\begin{itemize}
\item $\phi = \vect{U}\cdot\vx$ gives $\vu = \vect{U}$.  This is uniform
  flow with velocity $\vect{U}$ (e.g. flow in a straight pipe).
\item $\phi = (e^{\pm k z} \text{ or } \cosh kz \text{ or } \sinh kz)
  \times (e^{\pm \imath k x} \text{ or } \cos kx \text{ or } \sin kx)$
  gives a flow which is periodic in $x$ (e.g. waves).
\end{itemize}

\subsection*{Spherical polars}

The general axisymmetric solution of $\lapl\phi=0$ in spherical polars
is
\[
\phi = \sum_{n \ge 0} \left( A_n r^n + B_n r^{-n-1} \right) P_n(\cos \theta),
\]

where the $P_n$ are the Legendre polynomials.  We will only use the
first few modes.

\begin{itemize}
\item $\phi = - \frac{m}{4 \pi r}$ gives $\vu = \frac{m}{4 \pi}
  \frac{\Hat{\vect{r}}}{r^2}$.  This is a radial flow.  The total
  outflow over a sphere of radius $R$ is $4 \pi R^2 u_r = m$, which is
  independent of $R$ by mass conservation.  This is a \emph{point
    source} of strength $m$.  (If $m < 0$ it is a \emph{point sink}.)
\item $\phi = U r \cos \theta \equiv U z$ gives uniform flow again.
\item $\phi \propto r^{-2} \cos \theta$ gives \emph{dipole flow}.
\end{itemize}

\subsection*{Plane polars}

The general solution of $\lapl\phi=0$ in plane polars is

\[
\phi = K + A_0 \log r + B_0 \theta + \sum_{n \ge 1} \left(
A_n r^n + B_n r^{-n} \right) e^{\imath n \theta}.
\]

We again use only the first few modes.

\begin{itemize}
\item $\phi = \frac{m}{2\pi} \log r$ gives $\vu = \frac{m}{2 \pi}
  \frac{\Hat{\vect{r}}}{r}$.  This is a radial flow, a line source of
  strength $m$.
\item $\phi = \frac{\kappa \theta}{2 \pi}$ gives $\vu = \frac{\kappa}{2
    \pi} \frac{\Hat{\boldsymbol\theta}}{r}$.  The circulation about a
  circle of radius $R$ is $\kappa$, which is independent of $R$ by
  $\curl\vu=0$.  This is a \emph{line vortex} of circulation $\kappa$.
\item $\phi \propto r^{-1} \cos \theta$ is a 2D dipole.
\item $\phi \propto r^2 \cos \theta \equiv x^2 - y^2$ is a 2D
  straining flow.
\end{itemize}

\section{Applications}

\subsection{Uniform flow past a sphere}\label{sec:Usphere}

Consider a hard sphere of radius $a$ in a fluid having a uniform
velocity $U$ at infinity.

We have to solve the equations
\begin{align*}
\lapl\phi &= 0 &&\text{in $r > a$,} \\
\pd{\phi}{r} &= 0 &&\text{at $r=a$,} \\
\text{and } \phi &\sim U r \cos \theta &&\text{as $r \to \infty$.}
\end{align*}

We can satisfy the Laplace equation and boundary condition at infinity
with
\[
\phi = U \cos \theta \left( r + \frac{B}{r^2}\right).
\]

The boundary condition at $r=a$ yields $B = \frac{a^3}{2}$.  Thus

\[
\vu = \left(U \cos \theta \left(1 - \frac{a^3}{r^3} \right),
-U \sin \theta \left( 1 + \frac{a^3}{2 r^3} \right),0\right)
\]

in spherical polars $(r,\theta,\phi)$.

\vspace{4.5in}

\subsection{Uniform flow past a cylinder}

Consider a hard cylinder of radius $a$ in a fluid with a uniform
velocity $U$ at infinity and with a circulation $\kappa$.

We have to solve
\begin{align*}
\lapl\phi &= 0 &&\text{in $r > a$,} \\
\pd{\phi}{r} &= 0 &&\text{at $r=a$,} \\
\text{and } \phi &\sim U r \cos \theta &&\text{as $r \to \infty$.}
\end{align*}

We further need
\[
\oint_{r = a} \vu \cdot \vect{dl} = \kappa = \left[ \phi \right]_{r=a}
\]

to obtain a unique solution.  These conditions give
\[
\phi = U \cos \theta \left( r + \frac{a^2}{r} \right) + \frac{\kappa
  \theta}{2 \pi}.
\]

In plane polars $(r,\theta)$, we therefore have

\[
\vu = \left(U \cos \theta \left( 1 - \frac{a^2}{r^2} \right), - U \sin
  \theta \left( 1+ \frac{a^2}{r^2}\right) + \frac{\kappa}{2 \pi
    r}\right).
\]

\section[Pressure in irrotational flow]{The pressure in irrotational
  potential flow with potential forces}

The momentum and vorticity equations (\ref{eq:eumom}, \ref{eq:vorteq})
become
\[
\rho \left( \pd{\grad{\phi}}{t} + \dgrad{\vu}{\vu} \right) = -\grad
\left( p+ \chi \right)
\]
and
\[
\dgrad{\vu}{\vu} = \grad \left( \frac{1}{2} u^2 \right)
\]
respectively, and these combine to give
\[
\grad \left( \rho \pd{\phi}{t} + \frac{1}{2}\rho u^2 + p +\chi \right)
= 0,
\] which integrates to give
\begin{equation}\label{eq:bern2}
\rho \pd{\phi}{t} + \frac{1}{2}\rho u^2 + p +\chi = f(t) \text{
  independent of $\vx$.}
\end{equation}

\subsection*{Application}

\vspace{1.5in}

We can apply this theory to the free oscillations of a manometer.  We
need to calculate

\[
\phi(\vx,t) = \int_{\vx_0}^\vx \vu \cdot \vect{dl}.
\]

Let $s$ be the arc length from the bottom, with the equilibrium points at
$s = -l_1, l_2$.  From mass conservation the flow is uniform, so that
$u = \dot{h}$ everywhere.  Therefore $\phi = \dot{h} s$, and so
\[
\left.\pd{\phi}{t}\right|_\vx = \Ddot{h} s.
\]

At the interfaces the pressure is constant.  Using the equation for
potential flow \eqref{eq:bern2} we get
\[
\rho (-l_1 + h) \Ddot{h} + \tfrac{1}{2} \rho \Dot{h}^2 + p_a -
  \rho g h \sin \alpha = \rho (l_2 + h) + \tfrac{1}{2} \rho \Dot{h}^2
  + p_a + \rho g h \sin \beta.
\]

This simplifies to give
\[
\Ddot{h} = \frac{g (\sin \alpha + \sin \beta)}{l_1 + l_2} h,
\]

and so we have SHM (even for large oscillations).

\section{Bubbles}

\subsection{General theory for spherically symmetric motion}

The pressure is $p(r,t)$, and the far-field pressure is $p(\infty,t)$.
We have radial motion, $u \propto \frac{1}{r^2}$.  If the radius of
the bubble is $a(t)$, then $\dot{a} = u_r$ at $r=a$, which gives that
$\vu = \frac{\hat{\vect{r}}}{r^2}\dot{a}a^2 = \grad \phi$.  So $\phi =
- \frac{\dot{a}a^2}{r}$ and $\left.\pd{\phi}{t}\right|_r =
-\frac{\ddot{a}a^2 + 2\dot{a}^2 a}{r}$.  Putting all this together
gives
\[
-\rho \frac{\ddot{a}a^2 + 2\dot{a}^2 a}{r} +\frac{\rho}{2}
\frac{\dot{a}^2 a^4}{r^4} + p(r,t) = p(\infty,t).
\]
At $r=a$
\[
-\rho \ddot{a}a -\frac{3}{2} \rho \dot{a}^2 = p(\infty,t) - p(a,t),
\]
or
\[
\frac{\ud}{\ud t}\left( \frac{1}{2}\rho a^3 \dot{a}^2 \right) = a^2
\dot{a} \left( p(a,t) - p(\infty,t) \right).
\]
This can be interpreted as ``rate of change of kinetic energy equals
rate of working by pressure forces''.

Another rewrite gives
\[
p(r,t)-p(\infty,t) = \left( p(a,t) - p(\infty,t) \right)\frac{a}{r} +
\frac{1}{2} \dot{a}^2 \left( \frac{a}{r} -\frac{a^4}{r^4} \right)
\]

\subsection{Small oscillations of a gas bubble}

Assume $a(t) = a_0 + \delta a(t)$ and that $\delta_a(t)$ is small,
$p_\infty$ is constant in time and that the gas in the bubble has
pressure such that $\delta p_{\text{gas}} = - \gamma p_\infty \frac{3
  \delta_a}{a_0}$ (**which can be obtained from $P V^\gamma$ constant
for ideal adiabatic gas**).  Neglect surface tension.  Linearising
\[
-\rho \ddot{a}a -\frac{3}{2} \rho \dot{a}^2 = p(\infty,t) - p(a,t)
\]
about $a=a_0$ and $\dot{a} = 0$ we obtain
\[
-\rho a_0 \ddot{\delta_a} = \frac{3 \gamma p_\infty \delta_a}{a_0},
\]
which is SHM with $\omega = \left(\frac{3 \gamma p_\infty}{\rho
    a_o^2}\right)^{\frac{1}{2}}$.

\subsection{Total collapse of a void}

Bernoulli implies that the pressure decreases as the speed increases.
If this makes the pressure negative, then the liquid will break to
form a void filled only with vapour. This has important consequences
for valves and propellors (cavitation).

Consider a spherical void (ie $p(a,t) = 0$) at rest with $a = a_0$ and
$\dot{a} = 0$ at $t=0$ with a constant background pressure
$p_\infty$.  Now
\[
\frac{\ud}{\ud t}\left( \frac{1}{2}\rho a^3 \dot{a}^2 \right) = a^2
\dot{a} \left( p(a,t) - p(\infty,t) \right),
\]
so
\[
\frac{1}{2} \rho a^3 \dot{a}^2 = \frac{1}{3} p_\infty \left( a_0^3
  -a^3 \right).
\]

Integrate this again (numerically!), to get
\[
t_{\text{collapse}} = 0.92 \left( \frac{\rho a_0^2}{p_\infty}
\right)^{\frac{1}{2}}.
\]

\section[Translating Sphere]%
{Translating sphere \& inertial reaction to acceleration}

\subsection{Steady motion}

Use the inertial frame moving steadily with the sphere.  It was shown
in \S\ref{sec:Usphere} that uniform flow past a fixed sphere has
\[
\phi = U \cos \theta \left( r +\frac{a^3}{2r^2} \right) \text{ and}
\]
\[
\vu = \left( U \cos \theta \left(1 - \frac{a^3}{r^3} \right),-U \sin
  \theta \left( 1+ \frac{a^3}{2r^3}\right),0,\right).
\]
Hence at $r=a$, $\abs{\vu}=\frac{3}{2}U \sin \theta$.  Potential flow
means we can apply
\[
\rho \pd{\phi}{t} + \frac{1}{2}\rho u^2 + p +\chi \text{ independent
  of $\vx$,}
\] with $\pd{\phi}{t}=0$ (since steady motion) to compare $r=a$ and $r=\infty$.
This gives
\[
p(a,\theta)=p_\infty+\frac{1}{2}\rho U^2 \left( 1-\frac{9}{4} \sin^2
  \theta \right).
\]

This pressure distribution is symmetric fore and aft and around the
equator, so no force is exerted on a steadily moving sphere (**or
indeed any 3D body**).  This surprising result is called D'Alembert's
paradox.

\subsection[Effects of friction]{** Effects of friction **}

D'Alembert's paradox can be understood only by analogy with Newtonian
dynamics: in the absence of friction forces are needed only for
acceleration and not for uniform motion.

This potential flow result is a good approximation for bubbles
in steady motion, because they have slippery surfaces.  It is a bad
approximation for rigid spheres in steady motion.  What is observed in
experiments is \emph{separation} and a \emph{wake}.

\vspace{1.5in}

The small amount of friction produces a thin layer of fluid next to
the rigid surface that is slowed down from the potential flow.  This
thin \emph{boundary layer} is sensitive to the slowing down of the
surrounding potential flow from the equator to the rear stagnation
point, and detaches from the sphere into the body of the fluid to
produce a shear layer of concentrated vorticity (this is called
\emph{separation}).  Separation also occurs behind other bluff bodies
in steady translation but can be suppressed to a certain extent behind
streamlined/tapered bodies.  Boundary layers are covered in more
detail in Part IIB and Part III courses.

We can estimate the drag force as proportional to the projected area (A)
times the pressure difference: applying Bernoulli gives
\[
\text{drag force} = \tfrac{1}{2} C_D \rho U^2 A,
\]
where $C_D$ is a dimensionless coefficient that must be measured
experimentally (0.4 for a sphere, 1.1 for a disc, 1.0 for a cylinder).

\subsection{Accelerating spheres}

Potential flow is useful for slippery bubbles, rapid accelerations of
small (rigid) particles and small oscillations.

For the accelerating sphere, it is best to have the fluid at $\infty$
at rest.  Consider a sphere of radius $a$ and centre $\vx_0(t)$ and
velocity $\vu(t) = \dot{\vx}_0(t)$.  The velocity potential
problem is then
\begin{align*}
  \lapl\phi &= 0 \text{ in } r \ge a, \\
  \phi &\rightarrow 0 \text{ as } r \rightarrow \infty, \\
  \left.\pd{\phi}{r}\right|_t &= \vu(t)\cdot\vn \text{ on } r = a.
\end{align*}

with solution
\begin{align*}
  \phi &= - \frac{U \cos \theta a^3}{2r^2} \\
  &=- \frac{\vu\cdot\left(\vx - \vx_0(t) \right) a^3}{2 \abs{\vx -
      \vx_0(t)}^3}
\end{align*}

To calculate the force, we need
\[
\left.\pd{\phi}{t}\right|_r = \frac{-\dot{\vu}\cdot\vect{r} a^3}{2r^3}
+ \vu\cdot\left( \text{terms linear in $\dot{\vx}_0$} \right).
\]
Now $\grad{\phi} = \text{ terms linear in $\vu$}$.  These linear terms
must be the same as for steady motion.  Hence the pressure on the
sphere is $p = p_\infty + \rho \frac{\dot{\vu}\cdot\vect{r}}{2} +
\frac{1}{2} \rho U^2 \left( 1 - \frac{9}{4}\sin^2 \theta \right)$.
The force on the sphere is given by
\[
\int_{r=a} -p \vn \,\ud A = -\frac{\rho}{2} \int_{r=a} \left(
  \dot{\vu}\cdot\vect{r} \right) \frac{\vect{r}}{a} \,\ud A
\]
Now by the isotropy of the sphere, $\int_{r=a} r_i r_j \,\ud A =
\frac{4 \pi a^4}{3} \delta_{ij}$, so
\[
\vF = -\frac{1}{2} \rho \dot{\vu} \frac{4 \pi a^4}{3} \frac{1}{a} = -m^*
\dot{\vu}
\]
where $m^* = \frac{1}{2}\rho_{\text{fluid}} \frac{4 \pi a^3}{3}$, the added
(or effective or virtual) mass.

\subsection{Kinetic Energy}

The kinetic energy of fluid motion in a volume $V$ is

\begin{align*}
  T &= \int_V \frac{1}{2} \rho u^2 \,\ud V \\
  &= \frac{\rho}{2} \int_V \left( \grad{\phi} \right)^2 \,\ud V \\
  &= \frac{\rho}{2} \int_V \divr{\left(\phi \grad{\phi}\right)}
  - \phi \lapl\phi \,\ud V \\
  &= \frac{\rho}{2} \int_S \phi \left(\grad{\phi}\right)\cdot\vect{dS} \\
  &= \int_S \frac{\rho}{2} \phi \dgrad{\vn}{\phi}\, \ud S
\end{align*}

For the translating sphere we get $T = \frac{1}{2} m^* U^2$.

\section{Translating cylinders with circulation}

The potential for flow past a uniform cylinder with circulation
$\kappa$ is

\[
\phi = U \cos \theta \left( r + \frac{a^2}{r} \right) + \frac{\kappa
  \theta}{2 \pi},
\]
with
\[
\vu = \left( U \cos \theta \left( 1 - \frac{a^2}{r^2} \right), -U \sin
  \theta \left( 1 + \frac{a^2}{r^2} \right) + \frac{\kappa}{2 \pi r}
\right).
\]
On $r=a$, $\abs{\vu} = -2 \sin \theta U + \frac{\kappa}{2 \pi a}$, thus
\[
p(a,\theta) = p_\infty + \frac{1}{2} \rho \left( U^2 - \left(
    \frac{\kappa}{2 \pi a} - 2 U \sin \theta \right)^2 \right)
\]
and so (by doing the integral), $\vF = \left( 0,-\rho U \kappa \right)$
(in Cartesians). This is a lift force perpendicular to the velocity.
The same lift
force applies to arbitrary aerofoils (at least to the first
approximation).

\subsection[Generation of circulation]{** Generation of circulation
  **}

In order to calculate the lift on an aerofoil we thus need to know the
circulation $\kappa$.  Potential flow without circulation would have
very large velocities around the sharp trailing edge.

\vspace{1.5in}

This is strongly opposed by friction, and the flow avoids these high
velocities by shedding an eddy as the aerofoil starts to move.  This
eddy is of such a size as to streamline the flow on the aerofoil.

\vspace{1.5in}

The circulation around $C_1 + C_2$ is initially zero and so remains
zero at all times (by Kelvin's theorem).  Hence the circulation around
the aerofoil (the \emph{bound vortex}) is equal in magnitude to the
circulation in the shed \emph{starting vortex} (but opposite in
sign).  The strength is chosen to avoid a singularity at the sharp
trailing edge, a condition which gives $\kappa \approx \pi l U \sin
\alpha$, provided $\alpha < 14^\circ$.  For $\alpha > 14^\circ$ the
 flow separates and the aerofoil stalls.%
\footnote{See Fluids IIB or Acheson for more details.} 

Since $\divr\bso = 0$ vortex lines cannot end in the fluid, and in
fact the bound vortex on the wing is connected to the starting vortex
by two vortices shed from the wingtips.  These are responsible for the
observed vapour trails.

\vspace{1.5in}

\section{More solutions to Laplace's equation}

\subsection{Image vortices and 2D vortex dynamics}

Recall that a single line vortex has
\[
\phi = \frac{\kappa \theta}{2 \pi}, \text{ with } \vu = \frac{\kappa}{2
  \pi r} (-y,x).
\]

Since the potential flow equations are linear we can superpose
solutions.

\vspace{1.5in}

This configuration would give

\[
\phi = \sum_i \frac{\kappa_i \theta_i}{2 \pi}.
\]

This gives us a new method of finding solutions in ``nice''
geometries.  For instance, consider a vortex near a plane wall as
shown.

\vspace{2in}

The velocity field is the same as if there was an \emph{image vortex}
of opposite strength so that normal velocities cancel.  This image
produces a velocity $\frac{\kappa}{4 \pi d}$ at the real vortex, and
so the vortex trundles along parallel to the wall as shown.

\vspace{1.5in}

\subsubsection*{Application to dispersal of wingtip vortices}

\vspace{1.5in}

We have vortices at $(\pm x_0(t), y_0(t))$ and so we need to add
images at $(\pm x_0(t), -y_0(t))$.  Now

\[
\Dot{x}_0 = \frac{\kappa}{4 \pi} \frac{x_0^2}{y_0(x_0^2 + y_0^2)}
\quad
\text{ and }
\quad
\Dot{y}_0 = -\frac{\kappa}{4 \pi} \frac{y_0^2}{x_0(x_0^2 + y_0^2)}.
\]

This gives $\frac{\ud y_0}{\ud x_0} = - \frac{y_0^3}{x_0^3}$, or
\[
\frac{1}{x_0^2} + \frac{1}{y_0^2} = C.
\] 

\subsection{Flow in corners}

We use $\phi = r^\mu \sin \mu \theta$, which satisifies
\[
\pd{\phi}{\theta} = 0 \text{ on } \theta = \pm \frac{\pi}{2 \mu}.
\]

\vspace{2in}

Now $\abs{\vu} \propto r^{\mu - 1}$, and so there are infinite
velocities as $r \to 0$ if $\mu < 1$.  The effect of friction here is
to introduce a circulation.

\vspace{1.5in}

\chapter{Free Surface Flows}

\section{Governing equations}

The flow is assumed to start from rest and is thus irrotational (and
remains so).  Let the free surface be at $\zeta(x,y,t)$, $\abs{\zeta}$
sufficiently small for nice things to happen.  So

\begin{align*}
  \lapl\phi &= 0 \quad \zeta \ge z \ge -h, \\
  \rho \pd{\phi}{t} + \frac{1}{2}\rho \abs{\grad{\phi}}^2 + p +\rho g
  z
  &= f(t), \\
  p &= p_{\text{air}} \text{ at } z = \zeta, \\
  \pd{\zeta}{t} + \pd{\zeta}{x} \left.\pd{\phi}{x}\right|_{z = \zeta} +
  \pd{\zeta}{y} \left.\pd{\phi}{y}\right|_{z = \zeta}
  &= \left.\pd{\phi}{z}\right|_{z = \zeta}, \\
  \left.\pd{\phi}{z}\right|_{z=-h} &= 0.
\end{align*}

We will restrict to the 1-D case but even so, to have any hope of
solving this we have to linearise it.
\begin{enumerate}
\item Throw out the nonlinear terms.
\item Evaluate at $z=\zeta$ using information at $z=0$.
\item Throw out new nonlinear terms.
\end{enumerate}

\begin{align*}
  \lapl\phi &= 0 \quad 0 \ge z \ge -h, \\
  \rho \left. \pd{\phi}{t}\right|_{z=0} + \rho g \zeta &= f(t), \\
  p &= p_{\text{air}} \text{ at } z = \zeta, \\
  \pd{\zeta}{t} &= \left.\pd{\phi}{z}\right|_{z = 0}, \\
  \left.\pd{\phi}{z}\right|_{z=-h} &= 0.
\end{align*}

We seek separable solutions and obtain $\phi(x,z,t) = A \cosh k
\left(z+h \right) e^{i \left( k x - \omega t \right)}$.  Using the
dynamic boundary condition $\rho \left.\pd{\phi}{t}\right|_{z=0} +
\rho g \zeta = f(t)$, we obtain the dispersion relation
\[
\omega^2 = g k \tanh k h.
\]

In deep water, $h \gg \frac{1}{k}$, so $\tanh k h \approx 1$, giving
$\omega^2 = k h$ and $c = \sqrt{\frac{g}{k}}$\footnote{The wave
  speed...}.  In shallow water, $\tanh k h \approx k h$ giving $\omega
= k \sqrt{g h}$ and $c = \sqrt{g h}$.

\subsection{Particle paths under a wave}

By linearising, we get that
\begin{align*}
  x(t) &= x_0 - \frac{i a \cosh k \left(z_0 +h \right)}{\sinh k h}
  e^{i \left( k x_0 - \omega t
    \right)} \\
  z(t) &= z_0 + \frac{a \sinh k \left(z_0 +h \right)}{\sinh k h} e^{i
    \left( k x_0 - \omega t \right)},
\end{align*}

which is elliptic motion. $x_0$ and $z_0$ are the mean positions of
the particles in the $(x,z)$ plane.

In the deep water limit we get circular motion and in the shallow
water limit we get mainly horizontal motion.

\section{Standing Waves}

Consider waves in a deep rectangular box, $0 \le x \le a$, $0 \le y
\le b$ and $z \le 0$.  Look for linearised waves with displacement
$\zeta(x,y,t)$.  Try separable solutions to obtain
\[
\phi (x,y,z,t) = A \cos \frac{m \pi x}{s} \cos \frac{n \pi y}{b} e^{k
  z} e^{-i \omega t}.
\]
The Laplace equation determines $k$ from $m$ and $n$ by
\[
k^2 = \pi^2 \left( \frac{m^2}{a^2} + \frac{n^2}{b^2} \right),
\] and the dynamic boundary conditions determine $\omega^2 = g k$.  This is to
be expected, since a standing wave is the sum of progressive waves.

\subsection{Rayleigh-Taylor instability}

Turn the box in the previous section upside down (or equivalently
replace $g$ with $-g$).  This makes $\omega$ imaginary, leading to a
$e^{\sqrt{g k}t}$ term, which quickly magnifies any deviations from
$\zeta = 0$.  The largest growth rate is for large $k$, but this is
cancelled by surface tension in the real world.

\section{River Flows}

These are nonlinear problems, but can be solved since the shallowness
of the river in comparision to its length means that the flow is
nearly unidirectional.

\subsection{Steady flow over a bump}\label{sec:riverbump}

\vspace{1.5in}

What happens to the free surface at the bump?

We assume that

\begin{itemize}
\item the river has vertical sides and constant width;
\item it varies only slowly in the $x$ direction, so that the flow is
  (to a good approximation) horizontal and uniform across any vertical
  section (since $\pd{u}{z} = \omega_y = 0$);
\item the flow is steady;
\item far from the bump ($x \to \pm \infty$), $\zeta \to 0$, $h \to
  h_\infty$ and $U \to U_\infty$.
\end{itemize}

Then mass conservation gives
\begin{equation}\label{eq:mass1}
U (\zeta + h) = U_\infty h_\infty,
\end{equation}

whilst applying Bernoulli's equation \eqref{eq:bern1} to the surface
streamline gives

\begin{equation}\label{eq:freebern}
\frac{1}{2} \rho U^2 + p_{\text{atm}} + \rho g \zeta
= \frac{1}{2} \rho U_\infty^2 + p_{\text{atm}}.
\end{equation}

Now eliminate $\zeta$ between \eqref{eq:mass1} and \eqref{eq:freebern}
to obtain

\begin{equation}\label{eq:river1}
\frac{1}{2} U^2 + \frac{g U_\infty h_\infty}{U}
= \frac{1}{2} U_\infty^2 + g h.
\end{equation}

We can now extract information from \eqref{eq:river1} graphically.

\vspace{2in}

There are thus two roots and three possibilities.

\begin{itemize}
\item If the bump is too big then the assumption of steady/slowly
  varying flow must fail.  This gives a hydraulic jump --- see
  \S\ref{sec:hydraulic}.
\item If the bump is just right then the flow can pass smoothly from
  one root the other (e.g. flow over a weir in \S\ref{sec:weir}).
\item If the bump is not too large then the flow stays on the same
  root.  This has two subcases.
  \begin{itemize}
    \item On the left hand root: $U_\infty < \sqrt{g h_\infty}$ ---
      flow slower than shallow water waves --- we have
\[
\text{bump up} \Rightarrow h \downarrow \Rightarrow \text{RHS} \downarrow
\Rightarrow u \uparrow \Rightarrow \zeta \downarrow.
\]

i.e. slow deep flow converts PE to KE to maintain the mass flux. 

    \item On the right hand root: $U_\infty > \sqrt{g h_\infty}$ ---
      flow faster than shallow water waves --- we have
\[
\text{bump up} \Rightarrow h \downarrow \Rightarrow \text{RHS} \downarrow
\Rightarrow u \downarrow \Rightarrow \zeta \uparrow.
\]

i.e. fast shallow flow converts KE to PE to maintain the mass flux.

  \end{itemize}
\end{itemize}

Normal rivers are in the slow deep state.

\subsection{Flow out of a lake over a broad weir}\label{sec:weir}

This is the same as \S\ref{sec:riverbump} but with a smooth change of
branch.  We need the bump ``just right''.

\vspace{1.5in}

How fast is the outflow as a function of the minimum of $h(x)$?

The lake is large and deep, so we take the limit $h_\infty \to
\infty$, $U_\infty \to 0$ with $U_\infty h_\infty = Q$ fixed.  Mass
conservation gives

\[
U (\zeta + h) = Q
\]

and Bernoulli on the surface streamline gives
\[
\frac{1}{2} U^2 + g \zeta = 0.
\]

Eliminating $\zeta$ as before gives

\begin{equation}\label{eq:river2}
\frac{1}{2} U^2 + \frac{g Q}{U} = g h(x),
\end{equation}

which is a cubic for unknown $U(x)$ given $h(x)$.

\vspace{2in}

The flow starts high on the LH branch in the lake, but comes out of
the weir high on the RH branch.  The minimum of $h(x)$ (the crest of
the weir) must be at the join of the branches, and so
\[
Q = \left( \frac{8}{27} g h_{\text{min}}^3 \right)^{\frac{1}{2}},
\]

giving the flow rate as a function of the height of the weir.

At the crest of the weir,
$U = \left( \frac{2}{3} g h_{\text{min}}^3 \right)^{\frac{1}{2}}$ and
mass conservation gives
\[
\zeta_{\text{crest}} = - \frac{1}{3} h_{\text{min}}
\]
and the fluid depth is $\frac{2}{3} h_{\text{min}}$.  Also,
$U^2_{\text{crest}} = g ( \text{fluid depth} )_{\text{crest}}$, so
that the flow is travelling at the speed of shallow water waves at the
crest.  There can therefore be no communication from after the weir
back to the lake.

\subsection{Hydraulic jumps}\label{sec:hydraulic}

\vspace{2in}

Given $U_1$ and $h_1$ from river data and $h_2$ from tidal theory can
we predict the speed $V$ of the jump (and the flow $U_2$)?

The energy loss in the turbulent jump makes Bernoulli inapplicable
(friction is important in the small-scale unsteady motions).  However
the flat-bottomed case provides an application of the momentum
integral equation.

Change to a frame moving with the jump.  Away from the jump there is
no vertical acceleration and so the pressure is hydrostatic there.

\vspace*{2in}

Taking the horizontal component of

\[
\frac{\ud}{\ud t} \int_V \rho \vu \,\ud V = -\int_A \left(\rho \vu (\vu\cdot\vn)
+p \vn\right) \,\ud A + \int_V \vF^V \,\ud V,
\]

and noting that the flow is steady on average we obtain (from the area
integral)

\begin{multline*}
\rho (V + U_1)^2 h_1 + (p_a h_1 + \frac{1}{2} \rho g h_1^2) +
  p_a(h_2 - h_1)\\
 = \rho (V - U_2)^2 h_2 + (p_a h_2 + \frac{1}{2} \rho
  g h_2^2),
\end{multline*}

which boils down to
\[
(V+U_1)^2 h_1 - (V - U_2)^2 h_2 = \frac{1}{2} g (h_2^2 - h_1^2).
\]

Mass conservation gives $(V + U_1) h_1 = (V - U_2) h_2$, and
eliminating $V - U_2$ between these equations gives (eventually)

\[
(V+U_1)^2 = \frac{g (h_1+h_2)h_2}{2 h_1}.
\]

We can solve this for $V$ and then solve
\[
(V-U_2)^2 = \frac{g (h_1+h_2)h_1}{2 h_2}
\]
for $U_2$.

Note that if $h_2 > h_1$ then $V - U_2 < \sqrt{g h_2}$ and $V+U_1 >
\sqrt{g h_1}$ --- the jump travels faster than shallow water waves on
the river side and overtakes all information of its future arrival.

\backmatter

\begin{thebibliography}{9}
\bibitem{Acheson} D.J. Acheson, \emph{Elementary Fluid
    Dynamics}, OUP, 1990.
  
  {\sffamily \small This is an excellent book, easy to read and with
    everything in.  It is also good for Fluids IIB.  Highly
    recommended.}
  
\bibitem{Batchelor} G.K. Batchelor, \emph{An Introduction to Fluid
    Dynamics}, CUP, 1967.
  
  {\sffamily \small The lecturer recommended this, and it is a
    reasonably good book for Fluids IIB.  Personally I think you'd be
    wrong in your head to buy it for this course, but YMMV.}
  
\bibitem{VanDyke} M. van Dyke, \emph{An Album of Fluid Motion}, The
  Parabolic Press, 1982.
  
  {\sffamily \small Lots of pictures of flows.  An excellent book.  Go
    out and buy it.  Now.}
\end{thebibliography}

\section*{Related courses}

In Part IIA there are courses on \emph{Transport Processes} and
\emph{Theoretical Geophysics}.  The IIB fluids courses are \emph{Fluid
  Dynamics 2} and \emph{Waves in Fluid and Solid Media}, both of which
use the material in this course to some extent.

\end{document}